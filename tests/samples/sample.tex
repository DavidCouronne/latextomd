%!TEX encoding = UTF-8 Unicode
\documentclass[10pt]{article}
\usepackage[T1]{fontenc}
\usepackage[utf8]{inputenc} 
\usepackage{fourier}
\usepackage[scaled=0.875]{helvet}
\renewcommand{\ttdefault}{lmtt}
\usepackage{amsmath,amssymb,makeidx}
\usepackage[normalem]{ulem}
\usepackage{fancybox}
\usepackage{multirow}
\usepackage{tabularx}
\usepackage{ulem}
\usepackage{dcolumn}
\usepackage{textcomp}
\usepackage{diagbox}
\usepackage{lscape}
\newcommand{\euro}{\eurologo{}}
%Tapuscrit : Denis Vergès
\usepackage{pstricks,pst-plot,pst-text,pst-tree,pstricks-add}
\usepackage[left=3.5cm, right=3.5cm, top=3cm, bottom=3cm]{geometry}
\newcommand{\vect}[1]{\overrightarrow{\,\mathstrut#1\,}}
\newcommand{\R}{\mathbb{R}}
\newcommand{\N}{\mathbb{N}}
\newcommand{\D}{\mathbb{D}}
\newcommand{\Z}{\mathbb{Z}}
\newcommand{\Q}{\mathbb{Q}}
\newcommand{\C}{\mathbb{C}}
\renewcommand{\theenumi}{\textbf{\arabic{enumi}}}
\renewcommand{\labelenumi}{\textbf{\theenumi.}}
\renewcommand{\theenumii}{\textbf{\alph{enumii}}}
\renewcommand{\labelenumii}{\textbf{\theenumii.}}
\def\Oij{$\left(\text{O},~\vect{\imath},~\vect{\jmath}\right)$}
\def\Oijk{$\left(\text{O},~\vect{\imath},~\vect{\jmath},~\vect{k}\right)$}
\def\Ouv{$\left(\text{O},~\vect{u},~\vect{v}\right)$}
\usepackage{fancyhdr}
\usepackage[dvips]{hyperref}
\hypersetup{%
pdfauthor = {APMEP},
pdfsubject = {Corrigé du baccalauréat S},
pdftitle = {Asie 18 juin 2013},
allbordercolors = white,
pdfstartview=FitH}   
\thispagestyle{empty}
\usepackage[frenchb]{babel}
\usepackage[np]{numprint}
\begin{document}
\setlength\parindent{0mm}
\rhead{\textbf{A. P{}. M. E. P{}.}}
\lhead{\small Corrigé du baccalauréat S }
\lfoot{\small{Asie}}
\rfoot{\small{18 juin 2013}}
\renewcommand \footrulewidth{.2pt}
\pagestyle{fancy}
\thispagestyle{empty}

\begin{center}{\Large{\textbf{\decofourleft~Corrigé du baccalauréat S Asie 
18 juin 2013~\decofourright }}} \end{center}

\vspace{0,25cm}

%Dans l'ensemble du sujet, et pour chaque question, toute trace de recherche même incomplète, ou d'initiative même non fructueuse, sera prise en compte dans l'évaluation.
%
%\vspace{0,5cm}

\textbf{\textsc{Exercice 1} \hfill 5 points}

\textbf{Commun  à tous les candidats}

\medskip

%Dans cet exercice, les probabilités seront arrondies au centième. 
%
%\bigskip

\textbf{Partie A}

%\medskip
% 
%Un grossiste achète des boîtes de thé vert chez deux fournisseurs. Il achète 80\,\% de ses boîtes chez le fournisseur A et 20\,\% chez le fournisseur B.

\medskip
 
%10\,\% des boîtes provenant du fournisseur A présentent des traces de pesticides et 20\,\% de celles provenant du fournisseur B présentent aussi des traces de pesticides.
% 
%On prélève au hasard une boîte du stock du grossiste et on considère les évènements suivants : 
%
%\setlength\parindent{8mm}
%\begin{itemize}
%\item évènement A : \og la boîte provient du fournisseur A \fg{} ; 
%\item évènement B : \og la boîte provient du fournisseur B \fg{} ; 
%\item évènement S : \og la boîte présente des traces de pesticides \fg.
%\end{itemize}
%\setlength\parindent{0mm}
%
%\medskip
 
\begin{enumerate}
\item %Traduire l'énoncé sous forme d'un arbre pondéré.
Le grossiste a deux fournisseurs et il y a dans chaque boîte des traces de pesticides ou non. On a donc un arbre $2 \times 2$ :

\begin{center}
\pstree[treemode=R,nodesep=4pt]{\Tdot}
{
	\pstree{\TR{$A$}\taput{$0,8$}}
		{
\Tdot~[tnpos=r]{$S$}\taput{$0,1$}
\Tdot~[tnpos=r]{$\overline{S}$}\tbput{$0,9$}
 		}
	\pstree{\TR{$B$}\tbput{$0,2$}}
		{
\Tdot~[tnpos=r]{$S$}\taput{$0,2$}
 \Tdot~[tnpos=r]{$\overline{S}$}\tbput{$0,8$}
 		}
}
\end{center}
 
\item 
	\begin{enumerate}
		\item %Quelle est la probabilité de l'évènement $B \cap \overline{S}$ ?
En suivant la quatrième branche :
		
$p\left(B \cap \overline{S}\right)  = p(B) \times p_{B}\left(\overline{S} \right) = 0,2 \times 0,8 = 0,16$. 
		\item %Justifier que la probabilité que la boîte prélevée ne présente aucune trace de pesticides est égale à $0,88$.
On calcule de même :

$p\left(A \cap \overline{S}\right)  = p(A) \times p_{A}\left(\overline{S} \right) = 0,8 \times 0,9 = 0,72$.

$\{A~;~B\}$ étant une partition de l'univers, on a donc :

$p\left(\overline{S}\right) = p\left(A \cap \overline{S}\right)  + 	p\left(B \cap \overline{S}\right)  = 0,72 + 0,16 = 0,88$.
	\end{enumerate} 
\item Il faut donc calculer :
%On constate que la boîte prélevée présente des traces de pesticides. 

%Quelle est la probabilité que cette boîte provienne du fournisseur B ?


$p_{S}(B) = \dfrac{p(S \cap B)}{p(S)}$.

On a vu que $p\left(\overline{S}\right) = 0,88$, donc $p(S) = 1 - p\left(\overline{S}\right) = 0,12$.

Donc $p_{S}(B) =\dfrac{0,2 \times 0,2}{0,12} = \dfrac{4}{12} = \dfrac{1}{3} \approx 0,33$ au centième près.
\end{enumerate}

\bigskip
 
\textbf{Partie B}

\medskip
 
%Le gérant d'un salon de thé achète $10$~boîtes chez le grossiste précédent. On suppose que le stock de ce dernier est suffisamment important pour modéliser cette situation par un tirage aléatoire de $10$~boîtes avec remise.
% 
%On considère la variable aléatoire $X$ qui associe à ce prélèvement de $10$~boîtes, le nombre de boîtes sans trace de pesticides.
%
%\medskip
 
\begin{enumerate}
\item %Justifier que la variable aléatoire $X$ suit une loi binomiale dont on précisera les paramètres.
On a vu que la probabilité de tirer une boîte de façon aléatoire dans le stock du grossiste sans trouver de pesticides est égale à $0,88$. C'est une épreuve de Bernoulli.

Répéter de façon indépendante 10 fois cette expérience est donc une épreuve de Bernoulli de paramètres $n = 10$ et $p = 0,88$.

La variable $X$ suit donc une loi binomiale $\mathcal{B}(10~;~0,88)$.   
\item %Calculer la probabilité que les 10 boîtes soient sans trace de pesticides.
Il faut trouver $p(X = 10) = \binom{10}{10}\times 0,88^{10}\times (1 - 0,88)^{10 - 10} = 0,88^{10} \approx 0,28$ au centième près.
\item %Calculer la probabilité qu'au moins $8$~boîtes ne présentent aucune trace de pesticides.
Il faut trouver :

$p(X \geqslant 8) = p(X = 8) + p(X = 9) + p(X = 10) = $

$\binom{10}{8}\times 0,88^{8}\times (1 - 0,88)^{10 - 8} + \binom{10}{9}\times 0,88^{9}\times (1 - 0,88)^{10 - 9} + \binom{10}{10}\times 0,88^{10}\times (1 - 0,88)^{10 - 10} \approx$

$ \np{0,233043} + \np{0,379774} + \np{0,278501} \approx \np{0,891318} \approx 0,89$ au centième près
\end{enumerate}
 
\newpage
 
\textbf{Partie C}

\medskip
 
%¿ des fins publicitaires, le grossiste affiche sur ses plaquettes: \og 88\,\% de notre thé est garanti sans trace de pesticides \fg. 
%
%Un inspecteur de la brigade de répression des fraudes souhaite étudier la validité de l'affirmation. À cette fin, il prélève $50$~boîtes au hasard dans le stock du grossiste et en trouve $12$ avec des traces de pesticides.
%
%\medskip
% 
%On suppose que, dans le stock du grossiste, la proportion de boîtes sans trace de pesticides est bien égale à $0,88$.
% 
%On note $F$ la variable aléatoire qui, à tout échantillon de $50$~boîtes, associe la fréquence des boîtes ne contenant aucune trace de pesticides.

%\medskip
 
\begin{enumerate}
\item %Donner l'intervalle de fluctuation asymptotique de la variable aléatoire $F$ au seuil de 95\,\%.
On vérifie tout d'abord que :

$\bullet~~$ $n = 50$ et $50 \geqslant 30$ ;

$\bullet~~$ $np = 50 \times 0,88 = 44$ et $44 \geqslant 5$ ;

$\bullet~~$ $n(1 - p) = 50 \times 0,12 = 6$ et $6 \geqslant 5$.

On sait qu'alors l'intervalle de fluctuation asymptotique au seuil de 95\,\% est égale à :

I$_{f} = \left[0,88 - \dfrac{1,96 \times \sqrt{0,88 \times (1 - 0,88)}}{\sqrt{50}}~;~0,88 + \dfrac{1,96 \times \sqrt{0,88 \times (1 - 0,88)}}{\sqrt{50}}\right]$, d'où au centième près :

I$_{f} = [0,79~;~0,98]$. 
\item %L'inspecteur de la brigade de répression peut-il décider, au seuil de 95\,\%, que la publicité est mensongère ?
L'inspecteur de la brigade de répression constate une proportion de lots sans pesticides de 

$\dfrac{50 - 12}{50} \approx 0,76$.
Or $0,76 \notin \text{I}_{f}$, donc il doit constater au risque de 5\,\% que la publicité est mensongère. 
\end{enumerate}
 
\vspace{0,5cm}

\textbf{\textsc{Exercice 2} \hfill 6 points}

\textbf{Commun  à tous les candidats} 

%On considère les fonctions $f$ et $g$ définies pour tout réel $x$ par : 
%
%\[f(x) = \text{e}^x \quad  \text{et}\quad  g(x) = 1 - \text{e}^{- x}.\]
% 
%Les courbes représentatives de ces fonctions dans un repère orthogonal du plan, notées respectivement $\mathcal{C}_{f}$ et $\mathcal{C}_{g}$, sont fournies en annexe.

\bigskip
 
\textbf{Partie A}

\medskip
  
%Ces courbes semblent admettre deux tangentes communes. Tracer aux mieux ces tangentes sur la figure de l'annexe.
Voir la figure. 

\bigskip
 
\textbf{Partie B}

\medskip 

%Dans cette partie, on admet l'existence de ces tangentes communes. 
%
%On note $\mathcal{D}$ l'une d'entre elles. Cette droite est tangente à la courbe $\mathcal{C}_{f}$ au point A d'abscisse $a$ et tangente à la courbe $\mathcal{C}_{g}$ au point B d'abscisse $b$.
%
%\medskip
 
\begin{enumerate}
\item 
	\begin{enumerate}
		\item %Exprimer en fonction de $a$ le coefficient directeur de la tangente à la courbe $\mathcal{C}_{f}$ au point A.

Le coefficient directeur de la tangente à la courbe $\mathcal{C}_{f}$ au point A est égal à $f'(a)$. Or

$f'(x)  = \text{e}^{x}$, donc $f'(a) = \text{e}^{a}$. 
		\item %Exprimer en fonction de $b$ le coefficient directeur de la tangente à la courbe $\mathcal{C}_{g}$ au point B.
De même le coefficient directeur de la tangente à la courbe $\mathcal{C}_{g}$ au point B est égal à $g'(b)$. Or $g'(x) = - \left(- \text{e}^{-x}\right)$, donc  $g'(b) = \text{e}^{- b}$.
		\item %En déduire que $b = - a$.
Si les deux tangentes sont égales le coefficient directeur de leurs équations réduites sont égaux, soit :

$f'(a) = g'(b) \iff \text{e}^{a} = \text{e}^{- b}$ et par croissance de la fonction logarithme népérien :

$a = - b \iff b = - a$.
	\end{enumerate} 
\item %Démontrer que le réel $a$ est solution de l'équation 

%\[2( x -1)\text{e}^x + 1 = 0.\]
Une équation réduite de la tangente à la courbe $\mathcal{C}_{f}$ au point A est égale à :

$y - \text{e}^a = \text{e}^{a}(x - a) \iff y = x\text{e}^{a} + \text{e}^a(1  - a)$.

Une équation réduite de la tangente à la courbe $\mathcal{C}_{g}$ au point B est égale à :

$y - \left(1 - \text{e}^{- b}\right) = \text{e}^{- b}(x - b) \iff y = x\text{e}^{ - b} + 1 - \text{e}^{- b} - b\text{e}^{- b}$.

Ou en remplaçant $- b$ par $a$ :

$y = x\text{e}^{a} + 1 - \text{e}^{a} + a\text{e}^{a} \iff y = x\text{e}^{a} + 1 + \text{e}^{a}(a - 1)$.

Si les deux tangentes sont égales, leurs équations réduites sont les mêmes. On a déjà vu l'égalité des coefficients directeurs. Les ordonnées à l'origine sont aussi les mêmes soit :

$\text{e}^a(1  - a) = 1 + \text{e}^{a}(a - 1) \iff \text{e}^a(2  - 2a) = 1 \iff 2(a - 1)\text{e}^a + 1 = 0$.

Donc $a$ est solution de l'équation dans $\R$ :
\[2(x - 1)\text{e}^x + 1 = 0.\] 
\end{enumerate} 
\medskip
 
\textbf{Partie C}

\medskip
 
%On considère la fonction $\varphi$ définie sur  $\R$ par 
%
%\[\varphi(x) = 2(x - 1)\text{e}^x + 1.\]
 
\begin{enumerate}
\item 
	\begin{enumerate}
		\item %Calculer les limites de la fonction $\varphi$ en $- \infty$ et $+ \infty$.
		Sur $\R$, \: $\varphi(x) = 2x\text{e}^x - \text{e}^x + 1$.

On sait que $\displaystyle\lim_{x \to - \infty} \text{e}^x = 0$ et $\displaystyle\lim_{x \to - \infty} x\text{e}^x = 0$, d'où par somme de limite :

$\displaystyle\lim_{x \to - \infty} \varphi(x) = 1$.

La droite d'équation $y = 1$ est asymptote horizontale à la courbe représentative de $\varphi$.

On a $\displaystyle\lim_{x to + \infty} (x - 1) = + \infty$ et  $\displaystyle\lim_{x \to + \infty} \text{e}^x = + \infty$, d'où par somme de limites : $\displaystyle\lim_{x \to + \infty} \varphi(x) = + \infty$.
		\item %Calculer la dérivée de la fonction $\varphi$, puis étudier son signe.
Somme de fonctions dérivable sur $\R$, $\varphi$ est dérivable sur $\R$ et :

$\varphi'(x) = 2\text{e}^x + 2(x - 1)\text{e}^x = 2x\text{e}^x$.

Comme,  quel que soit $x \in \R$; \: $\text{e}^x > 0$, le signe de $\varphi'(x)$ est celui de $x$. Donc sur $]- \infty~;~0[,$

$\varphi'(x) < 0$ : la fonction est décroissante sur cet intervalle et sur $]0~;~+ \infty[$, \:$\varphi'(x) > 0$ : la fonction $\varphi$ est croissante sur cet intervalle. D'où le tableau de variations :  
		\item %Dresser le tableau de variation de la fonction $\varphi$ sur $\R$. Préciser la valeur de $\varphi(0)$.
		~
\psset{unit=1cm}
\begin{center}
\begin{pspicture}(7,3)
\psframe(7,3)\psline(0,2)(7,2)\psline(0,2.5)(7,2.5)
\psline(1,0)(1,3)
\uput[u](0.,2.5){$x$} \uput[u](1.5,2.5){$- \infty$} \uput[u](4,2.5){$0$} \uput[u](6.5,2.5){$ + \infty$}
\rput(0.5,2.25){$f'(x)$}\rput(2.5,2.25){$-$}\rput(4,2.25){$0$}\rput(5.5,2.25){$+$}
\psline{->}(1.5,1.5)(3.5,0.5)
\psline{->}(4.5,0.5)(6.5,1.5) 
\uput[d](1.2,2){1}\uput[d](6.5,2){$+ \infty$}
\uput[u](4,0){$- 1$}\rput(0.5,1){$f(x)$}
\end{pspicture}
\end{center}
	\end{enumerate} 
\item
	\begin{enumerate}
		\item %Démontrer que l'équation $\varphi(x) = 0$ admet exactement deux solutions dans $\R$.
Sur $]- \infty~;~0]$ la fonction $\varphi$ est continue et strictement décroissante à valeurs dans $[- 1~;~1]$. Comme $0 \in  [- 1~;~1]$ il existe un réel unique $\alpha$ de $]- \infty~;~0]$ tel que $f(\alpha) = 0$.

Le même raisonnement sur l'intervalle $[0~;~+\infty[$ montre qu'il existe un réel unique de cet intervalle $\beta$ tel que $f(\beta) = 0$.

Donc l'équation $\varphi(x) = 0$ admet exactement deux solutions dans $\R$. 
		\item %On note $\alpha$ la solution négative de l'équation $\varphi(x) = 0$ et $\beta$ la solution positive de cette équation.
		 
%À l'aide d'une calculatrice, donner les valeurs de $\alpha$ et $\beta$ arrondies au centième.
La calculatrice donne successivement :

$\varphi(- 2) \approx 0,18$ et $\varphi(- 1) \approx -0,47$, donc $- 2 < \alpha < - 1$ ;

$\varphi(- 1,7) \approx 0,013$ et $\varphi(- 1,6) \approx -0,05$, donc $- 1,7 < \alpha < - 1,6$ ;

$\varphi(- 1,68) \approx 0,001$ et $\varphi(- 1,67) \approx -0,005$, donc $- 1,68 < \alpha < - 1,67$ ;

$\varphi(- 1,679) \approx 0,00041$ et $\varphi(- 1,678) \approx -0,0002$, donc $- 1,679 < \alpha < - 1,678$.

Conclusion au centième près $\alpha \approx - 1,68$.

De la même façon on obtient $\beta \approx 0,77$. 
	\end{enumerate} 
\end{enumerate}

\bigskip
 
\textbf{Partie D}

\medskip

%Dans cette partie, on démontre l'existence de ces tangentes communes, que l'on a admise dans la partie B.
% 
%On note E le point de la courbe $\mathcal{C}_{f}$ d'abscisse $\alpha$ et F le point de la courbe $\mathcal{C}_{g}$ d'abscisse $- \alpha$ ($a$ est le nombre réel défini dans la partie C).

%\medskip
 
\begin{enumerate}
\item %Démontrer que la droite (EF) est tangente à la courbe $\mathcal{C}_{f}$ au point E.
Le coefficient directeur de la tangente en E à $\mathcal{C}_{f}$ est $\text{e}^{\alpha}$.

Le coefficient directeur de la droite (EF) est: $\dfrac{1 - \text{e}^{\alpha} - \text{e}^{\alpha}}{- \alpha - \alpha} = \dfrac{1 - 2\text{e}^{\alpha}}{- 2\alpha}$.

Or $\alpha$ est solution de l'équation : $2(x - 1)\text{e}^x + 1 = 0$, autrement dit 

$2(\alpha - 1)\text{e}^{\alpha} + 1 = 0 \iff 2\alpha\text{e}^{\alpha} = 2\text{e}^{\alpha} - 1$, d'où en revenant au coefficient directeur de la droite (EF) : $\dfrac{1 - 2\text{e}^{\alpha}}{- 2\alpha} = \dfrac{- 2 \alpha\text{e}^{\alpha}}{-2\alpha} = \text{e}^{\alpha}$

Conclusion : la droite (EF) est bien la tangente à la courbe $\mathcal{C}_{f}$ au point d'abscisse $\alpha$ et la tangente à la courbe $\mathcal{C}_{g}$ au point d'abscisse $-\alpha$.

\item %Démontrer que (EF) est tangente à $\mathcal{C}_{g}$ au point F.
Le coefficient directeur de la tangente à la courbe $\mathcal{C}_{g}$ au point d'abscisse $- \alpha$ est $\text{e}^{- (- \alpha)} = \text{e}^{\alpha}.$

On a vu dans la question précédente que la droite (EF) a pour coefficient directeur $\text{e}^{\alpha}$ et contient le point F.

Conclusion la droite (EF) est la tangente à la courbe $\mathcal{C}_{g}$ au point d'abscisse $- \alpha$. 
\end{enumerate}

\vspace{0,5cm}

\textbf{\textsc{Exercice 3} \hfill 4 points}

\textbf{Commun  à tous les candidats}

\medskip

%\emph{Les quatre questions de cet exercice sont indépendantes.\\ 
%Pour chaque question, une affirmation est proposée. Indiquer si chacune d'elles est vraie ou fausse, en justifiant la réponse. Une réponse non justifiée ne rapporte aucun point.}
%
%\medskip
% 
%Dans les questions 1. et 2., le plan est rapporté au repère orthonormé direct \Ouv. 
%
%On considère les points A, B, C, D et E d'affixes respectives : 
%
%\[a = 2 + 2\text{i},\quad  b = - \sqrt{3} + \text{i},\quad c = 1 + \text{i}\sqrt{3},\quad d = - 1 + \dfrac{\sqrt{3}}{2}\text{i}\quad \text{et}\quad e = - 1 + \left(2 + \sqrt{3} \right)\text{i}.\] 

\begin{enumerate}
\item \textbf{Affirmation 1} : VRAIE%les points A, B et C sont alignés.

On a $\vect{\text{AB}}\left(- \sqrt{3} - 2~;~- 1\right)$ et $\vect{\text{AC}}\left(- 1~;~\sqrt{3} - 2\right)$.

D'où $\vect{\text{AC}} = \left(2 - \sqrt{3}\right)\vect{\text{AB}}$.

Les vecteurs sont colinéaires donc les points A, B et C sont alignés.
\item \textbf{Affirmation 2} : FAUSSE % les points B, C et D appartiennent à un même cercle de centre E.

On calcule successivement :

EB$^2  = 8 ~; ~ \text{EC}^2 = 8$ et ED$^2
 = \dfrac{19}{4} + 2\sqrt{3} \neq 8$.
 
 Les points B, C et D ne sont pas équidistants de E.
\item %Dans cette question, l'espace est muni d'un repère \Oijk.
 
%On considère les points I(1~;~0~;~0), J(0~;~1~;~0) et K(0~;~0~;~1).  

\textbf{Affirmation 3} : VRAIE %la droite $\mathcal{D}$ de représentation paramétrique $\left\{\begin{array}{l c l}
%x &=& 2 - t \\
%y &=& 6 - 2 t\\
%z &=&- 2 + t
%\end{array}\right.$  où $t \in \R$, coupe le plan  (IJK) au point E$\left(- \dfrac{1}{2}~;~1~;~\dfrac{1}{2} \right)$.

Une équation du plan (IJK) est $x + y + z = 1$. Un  point commun à ce plan et à la droite $\mathcal{D}$ a ses coordonnées telles que :

$2 - t + 6 - 2t - 2 + t = 1 \iff 5 = 2t \iff t = \dfrac{5}{2}.$

Ce point commun existe donc et a pour coordonnées $\left(- \dfrac{1}{2}~;~1~;~\dfrac{1}{2} \right)$.
\item %Dans le cube ABCDEFGH, le point T est le milieu du segment [HF].

%\medskip
%
%\begin{center}
%\psset{unit=1cm}
%\begin{pspicture}(8,7.4)
%\psline(2.1,1)(6.2,1)(5.6,5)(1.5,5)(2.6,6.9)(6.8,6.9)(5.6,5)%ABFEHGF
%\psline(6.2,1)(7.4,3)(6.8,6.9)%BCG
%\psline(0,5.42)(1.5,5)
%\psline(7.4,3)(8.6,2.7)
%\psline(2.1,1)(1.5,5)
%\psline[linestyle=dotted,linewidth=1.5pt](1.5,5)(7.4,3)%EC
%\psline(4.8,7.6)(4.1,6)
%\psline(2.1,1)(1.8,0.4)
%\psline[linestyle=dotted,linewidth=1.5pt](4.1,6)(2.1,1)
%\psline[linestyle=dotted,linewidth=1.5pt](2.1,1)(3.3,3)(7.4,3)
%\psline[linestyle=dotted,linewidth=1.5pt](3.3,3)(2.6,6.9)
%\uput[l](2.1,1){A} \uput[dr](6.2,1){B} \uput[ur](7.4,3){C} 
%\uput[ur](3.3,3){D} \uput[ul](1.5,5){E} \uput[ul](5.6,5){F} 
%\uput[ur](6.8,6.9){G} \uput[ul](2.6,6.9){H} \uput[ul](4.1,6){T} 
%\end{pspicture}
%\end{center} 


\textbf{Affirmation 4} : VRAIE %les droites (AT) et (EC) sont orthogonales.

(EFGH) est un carré donc le milieu T de [HF] est le milieu de [EG].

On a donc $\vect{\text{ET}} = \dfrac{1}{2}\vect{\text{EG}}$.

En prenant par exemple le repère $\left(\text{A},~\vect{\text{AB}}~;~\vect{\text{AD}}~;~\vect{\text{AE}}\right)$ calculons le produit scalaire :

$\vect{\text{AT}} \cdot \vect{\text{EC}} = \left(\vect{\text{AE}} + \vect{\text{ET}} \right) \cdot \left(\vect{\text{EG}} + \vect{\text{GC}} \right) = 
\left(\vect{\text{AE}} + \dfrac{1}{2}\vect{\text{EG}} \right) \cdot \left(\vect{\text{EG}} + \vect{\text{GC}} \right) =$
 
$\vect{\text{AE}}\cdot \vect{\text{EG}} + \vect{\text{AE}} \cdot \vect{\text{GC}} + \dfrac{1}{2}\vect{\text{EG}} \cdot \vect{\text{EG}} + \dfrac{1}{2}\vect{\text{EG}}\cdot\vect{\text{GC}}$.

Or ABCDEFGH est un cube, donc $\vect{\text{AE}} \cdot \vect{\text{EG}} = 0$ et $\vect{\text{EG}} \cdot \vect{\text{GC}} = 0$.

De plus $\vect{\text{AE}} = - \vect{\text{GC}}$ et EG $ = c\sqrt{2}$, $c$ étant la mesure du côté du cube.

Finalement : $\vect{\text{AT}} \cdot \vect{\text{EC}} = - c^2  + \dfrac{1}{2}\left(c \sqrt{2} \right)^2 = - c^2 + c^2 = 0$.

Les vecteurs sont orthogonaux donc les droites (AT) et (EC) sont orthogonales.
\end{enumerate}

\vspace{0,5cm}

\textbf{\textsc{Exercice 4} \hfill 5 points}

\textbf{Candidats n'ayant pas choisi l'enseignement de spécialité}  

\bigskip

\textbf{Partie A}

\medskip
 
%On considère la suite $\left(u_{n}\right)$ définie par : $u_{0} = 2$ et, pour tout entier naturel $n$ : 
%
%\[u_{n+1} = \dfrac{1 + 3u_{n}}{3 + u_{n}}.\] 
% 
%On admet que tous les termes de cette suite sont définis et strictement positifs.
%
%\medskip
 
\begin{enumerate}
\item %Démontrer par récurrence que, pour tout entier naturel $n$, on a : $u_{n} > 1$.
\emph{Initialisation} : la relation est vraie au rang $0$ ;

\emph{Hérédité} : supposons que pour tout naturel $p$ tel que $u_{p} > 1$.

$\dfrac{1 + 3u_{p}}{3 + u_{p}} = \dfrac{3 + u_{p} - 2 + 2u_{p}}{3 + u_{p}} =  \dfrac{\left(3 + u_{p}\right) + \left(2u_{p} - 2\right)}{3 + u_{p}} = 1  + 2\dfrac{u_{p} - 1}{3 + u_{p}}$.

Par hypothèse de récurrence on a :

$u_{p} - 1$ et comme $u_{p} > 1,\, 3 + u_{p} > 4 > 0$ donc son inverse $\dfrac{1}{3 + u_{p}} > 0$ et finalement $\dfrac{u_{p} - 1}{3 + u_{p}} > 0$, c'est-à-dire que $u_{p+1} = \dfrac{1 + 3u_{p}}{3 + u_{p}} > 1$

Conclusion : la propriété est vraie au rang $0$, et elle est héréditaire à partir de tout rang, donc d'après le principe de récurrence,   pour tout entier naturel $n$, \, $u_{n} > 1$.
\item  
	\begin{enumerate}
		\item %Établir que, pour tout entier naturel $n$, on a : $u_{n+1}- u_{n} = \dfrac{\left(1 - u_{n} \right)\left(1 + u_{n} \right)}{3+ u_{n}}$.
Quel que soit le naturel $n$, \, $u_{n+1}- u_{n} = \dfrac{1 + 3u_{n}}{3 + u_{n}} - u_{n} = \dfrac{1 + 3u_{n} - 3u_{n}- u_{n}^2}{3 + u_{n}} = \dfrac{1 - u_{n}^2}{3 + u_{n}} = \dfrac{\left(1 - u_{n} \right)\left(1 + u_{n} \right)}{3+ u_{n}}$.	
		\item %Déterminer le sens de variation de la suite $\left(u_{n}\right)$. 
On sait que quel que soit le naturel $n$, \, $u_{n} > 1 \Rightarrow u_{n}^2 > 1^2 \Rightarrow 1 - u_{n}^2 < 0$ et comme $3 + u_{n} > 0$ et finalement $u_{n+1}  - u_{n} < 0$ ce qui signifie que la suite $\left(u_{n}\right)$ est décroissante.

  %En déduire que la suite $\left(u_{n}\right)$ converge.
La suite $\left(u_{n}\right)$ est décroissante et minorée par $1$ : elle converge vers une limite supérieure ou égale à $1$. 
	\end{enumerate}
\end{enumerate}
	
\bigskip

\textbf{Partie B}

\medskip

%On considère la suite $\left(u_{n}\right)$ 	définie par : $u_{0} = 2$ et, pour tout entier nature $n$ :
%
%\[ u_{n+1} = \dfrac{1 + 0,5u_{n}}{0,5 + u_{n}}.\]
% 
%On admet que tous les termes de cette suite sont définis et strictement positifs.
%
%\medskip
 
\begin{enumerate}
\item 

%On considère l'algorithme suivant :
%\begin{center}
%\begin{tabular}{|c |l|}\hline
% Entrée& Soit un entier naturel non nul $n$\\ \hline 
%Initialisation &Affecter à $u$ la valeur 2\\ \hline 
%\multirow{4}{1.2cm}{Traitement et sortie }&POUR $i$ allant de 1 à $n$\\ 
%&\hspace{1cm}Affecter à $u$ la valeur $\dfrac{1 + 0,5u}{0,5 + u}$\\  
%&\hspace{1cm}Afficher $u$\\ \hline 
%&FIN POUR\\ \hline
%\end{tabular}
%\end{center}
% 
%Reproduire et compléter le tableau suivant, en faisant fonctionner cet algorithme pour $n = 3$. Les valeurs de $u$ seront arrondies au millième. 

\begin{center}
\begin{tabularx}{0.6\linewidth}{|*{4}{>{\centering \arraybackslash}X|}}\hline 
$i$&1&2& 3\\ \hline 
$u$&0,800&1,077&0,976\\ \hline 
\end{tabularx}
\end{center} 
\item %Pour $n = 12$, on a prolongé le tableau précédent et on a obtenu : 
%
%\begin{center}
%\begin{tabularx}{\linewidth}{|c|*{9}{>{\centering \arraybackslash}X|}}\hline 
%$i$&4&5&6&7&8&9&10&11&12\\ \hline
%$u$&\footnotesize\np{1,0083}&\footnotesize\np{0,9973}&\footnotesize\np{1,0009}&\footnotesize\np{0,9997}&\footnotesize\np{1,0001}&\footnotesize \np{0,99997}&\footnotesize\np{1,00001}&\footnotesize \np{0,999996}&\footnotesize \np{1,000001}\\ \hline
%\end{tabularx}
%\end{center}
%
%Conjecturer le comportement de la suite $\left(u_{n}\right)$ à l'infini.
Il semble que la suite converge vers $1$ par valeurs alternativement supérieures et inférieures. 
\item %On considère la suite $\left(v_{n}\right)$ définie, pour tout entier naturel $n$, par : $v_{n} = \dfrac{u_{n} - 1}{u_{n} + 1}$. 
	\begin{enumerate}
		\item %Démontrer que la suite $\left(v_{n}\right)$ est géométrique de raison $- \dfrac{1}{3}$.
$V_{n+1} = \dfrac{u_{n+1} - 1}{u_{n+1} + 1} = \dfrac{\frac{1 + 0,5u_{n}}{0,5 + u_{n}} - 1}{\frac{1 + 0,5u_{n}}{0,5 + u_{n}} + 1}  = \dfrac{0,5 - 0,5u_{n}}{1,5 + 1,5u_{n}} = \dfrac{- 0,5\left(u_{n} - 1\right)}{1,5\left(u_{n} + 1 \right)} = -\dfrac{1}{3}v_{n}$.

La suite $\left(v_{n}\right)$ est donc géométrique de raison $- \dfrac{1}{3}$.
		\item %Calculer $v_{0}$ puis écrire $v_{n}$ en fonction de $n$.
On a $v_{0} = \dfrac{2 - 1}{2 + 3} = \dfrac{1}{3}$.

On sait qu'alors pour tout naturel $n,\, v_{n} = \dfrac{1}{3}\times \left(- \dfrac{1}{3} \right)^n$.
	\end{enumerate} 
\item
	\begin{enumerate}
		\item %Montrer que, pour tout entier naturel $n$, on a : $v_{n} \neq 1$.
		Quel que soit le naturel $n$, \, $\left(- \dfrac{1}{3} \right)^n \leqslant 1$, donc $v_{n} \leqslant \dfrac{1}{3}$ et par conséquent $v_{n}\neq 1$.  
		\item %Montrer que, pour tout entier naturel $n$, on a : $u_{n} = \dfrac{1 + v_{n}}{1 - v_{n}}$.
$v_{n} = \dfrac{u_{n} - 1}{u_{n} + 1} \iff v_{n}\left(u_{n} + 1 \right) = u_{n} - 1 \iff 
v_{n}u_{n} + v_{n} = u_{n} - 1 \iff$

$ v_{n}u_{n} - u_{n}+  =  - 1  - v_{n} \iff u_{n}\left(v_{n} - 1\right) = - 1  - v_{n}$ et comme $v_{n} \neq 1$, 

$u_{n} = \dfrac{- 1 - v_{n}}{v_{n} - 1} = \dfrac{1 + v_{n}}{1 - v_{n}}$.  
		\item %Déterminer la limite de la suite $\left(u_{n}\right)$.
Comme $- 1 < - \dfrac{1}{3} < 1$, on sait que $\displaystyle \lim_{n \to + \infty} \left(- \dfrac{1}{3} \right)^n = 0$, soit  $\displaystyle \lim_{n \to + \infty} v_{n} = 0$, donc d'après le résultat précédent $\displaystyle \lim_{n \to + \infty} u_{n} = \dfrac{1}{1} = 1$.
	\end{enumerate}
\end{enumerate}

\vspace{0,5cm}

\textbf{\textsc{Exercice 4} \hfill 5 points}

\textbf{Candidats ayant choisi l'enseignement de spécialité} 

\medskip
 
%Un logiciel permet de transformer un élément rectangulaire d'une photographie.
% 
%Ainsi, le rectangle initial OEFG est transformé en un rectangle OE$'$F$'$G$'$, appelé image de OEFG.
%
%\begin{center}
%\psset{unit=1cm}
%\begin{pspicture}(7.5,6.5)
%\pspolygon(3.3,0.8)(7,4.8)(5.4,6.25)(1.75,2.2)
%\pspolygon(3.3,0.8)(5.1,2.75)(2.1,5.6)(0.3,3.5)
%\uput[dr](5.1,2.8){E} \uput[u](2.1,5.6){F} \uput[ul](0.3,3.5){G} 
%\uput[d](3.3,0.8){O} \uput[dr](7,4.8){E$'$} \uput[u](5.4,6.25){F$'$} 
%\uput[dl](1.75,2.2){G$'$}
%\rput(3,0.12){Figure 1} 
%\end{pspicture}
%\end{center} 
% 
%L'objet de cet exercice est d'étudier le rectangle obtenu après plusieurs transformations successives.
%
%\bigskip
 
\textbf{Partie A}

\medskip
 
%Le plan est rapporté à un repère orthonormé \Oij.
% 
%Les points E, F et G ont pour coordonnées respectives (2~;~2), $(-1~;~5)$ et $(-3~;~3)$.
% 
%La transformation du logiciel associe à tout point $M(x~;~y)$ du plan le point $M'(x'~;~y')$, image du point $M$ tel que: 
%
%\renewcommand\arraystretch{1.8}
%\[\left\{\begin{array}{l c l}
%x'&=&\dfrac{5}{4}x + \dfrac{3}{4}y\\
%y'&=&\dfrac{3}{4}x + \dfrac{5}{4}y
%\end{array}\right.\]
%\renewcommand\arraystretch{1}
%
%\begin{center}
%\psset{unit=1cm}
%\begin{pspicture}(-4,-2)(3,5.5)
%\psaxes[linewidth=1.5pt,Dx=10,Dy=10]{->}(0,0)(-4,-1)(3,5.5)
%\pspolygon(0,0)(2,2)(-1,5)(-3,3)
%\uput[dl](0,0){O} \uput[ur](2,2){E} \uput[u](-1,5){F} \uput[l](-3,3){G}
%\rput(-1,-1.5){Figure 2} 
%\end{pspicture}
%\end{center}

\begin{enumerate}
\item 
	\begin{enumerate}
		\item %Calculer les coordonnées des points E$'$, F$'$ et G$'$, images des points E, F et G par cette transformation.
On a :

$\left\{\begin{array}{l c l}
x_{\text{E}'}&=&\frac{5}{4}\times 2 + \frac{3}{4}\times 2\\
y_{\text{E}'}&=&\frac{3}{4}\times 2 + \frac{5}{4}\times 2\\
\end{array}\right. \iff \left\{\begin{array}{l c l}
x_{\text{E}'}&=& 4\\
y_{\text{E}'}&=&4
\end{array}\right.$

$\left\{\begin{array}{l c l}
x_{\text{F}'}&=&\frac{5}{4}\times (- 1) + \frac{3}{4}\times 5\\
y_{\text{F}'}&=&\frac{3}{4}\times (- 1) + \frac{5}{4}\times 5\\
\end{array}\right. \iff \left\{\begin{array}{l c l}
x_{\text{F}'}&=& \frac{5}{2}\\
y_{\text{F}'}&=&\frac{11}{2}
\end{array}\right.$

$\left\{\begin{array}{l c l}
x_{\text{G}'}&=&\frac{5}{4}\times (- 3) + \frac{3}{4}\times 3\\
y_{\text{G}'}&=&\frac{3}{4}\times (- 3) + \frac{5}{4}\times 3\\
\end{array}\right. \iff \left\{\begin{array}{l c l}
x_{\text{G}'}&=&- \frac{3}{2}\\
y_{\text{G}'}&=&\frac{3}{2}
\end{array}\right.$
		\item %Comparer les longueurs OE et OE$'$ d'une part, OG et OG$'$ d'autre part.
OE$^2 = 2^2 + 2^2 = 8$, donc OE $ = 2\sqrt{2}$.

OE$'^2 	 = 4^2 + 4^2 = 32$, donc OE$' = 4\sqrt{2}$. Donc OE$' = 2 \text{OE}$.

OG$^2 = (- 3)^2 + 3^2 = 9 + 9 = 18$, donc OG $ = 3\sqrt{2}$ ;

OG$'^2 = \left(- \frac{3}{2} \right)^2 + \left(\frac{3}{2} \right)^2 = \frac{18}{4}$, donc OG$' = \frac{3\sqrt{2}}{2}$. Donc OG
$' = \dfrac{1}{2}\text{OG}$.
	 
%Donner la matrice carrée d'ordre 2, notée $A$, telle que: $\begin{pmatrix}x'\\y' \end{pmatrix}= A \begin{pmatrix}x\\y \end{pmatrix}$.
On a $A = \begin{pmatrix}\frac{5}{4}&\frac{3}{4}\\\frac{3}{4}&\frac{5}{4} \end{pmatrix}$.
	\end{enumerate}
\end{enumerate}

\bigskip
 
\textbf{Partie B}

\medskip
 
%Dans cette partie, on étudie les coordonnées des images successives du sommet F du rectangle OEFG lorsqu'on applique plusieurs fois la transformation du logiciel.
 
\begin{enumerate}
\item %On considère l'algorithme suivant destiné à afficher les coordonnées de ces images successives.
 
%Une erreur a été commise.
% 
%Modifier cet algorithme pour qu'il permette d'afficher ces coordonnées.
%
%\begin{center}
%\begin{tabular}{|c|l|}\hline 
%Entrée &Saisir un entier naturel non nul $N$\\ \hline 
%\multirow{2}{2cm}{Initialisation }&Affecter à $x$ la valeur $- 1$\\ 
%&Affecter à $y$ la valeur 5\\ \hline 
%\multirow{6}{2cm}{Traitement}&POUR $i$ allant de 1 à $N$\\ 
%&Affecter à $a$ la valeur $\frac{5}{4} x + \frac{3}{4}y$\\ 
%&Affecter à $b$ la valeur $\frac{3}{4}x + \frac{5}{4}y$\\ 
%&Affecter à $x$ la valeur $a$\\ 
%&Affecter à $y$ la valeur $b$\\ 
%&FIN POUR\\ \hline 
%Sortie &Afficher $x$, afficher $y$\\ \hline
%\end{tabular}
%\end{center} 
Il suffit d'écrire avant le FIN POUR : afficher $x$, afficher $y$
\item %On a obtenu le tableau suivant :

%\[\begin{array}{|*{8}{c|}} \hline
%i &1 &2 &3 &4 &5 &10 &15\\ \hline 
%x &2,5 &7,25 &15,625 &\np{31,8125} &\np{63,9063} &\np{2047,9971} &\np{65535,9999}\\ \hline 
%y &5,5 &8,75 &16,375 &\np{32,1875} &\np{64,0938} &\np{2048,0029} &\np{65536,0001}\\ \hline
%\end{array}\]
% 
%Conjecturer le comportement de la suite des images successives du point F.
Il semble que les cordonnées sont de plus en plus grandes tout en se rapprochant (les points images sont de plus en plus proches de la droite $y = x$.) 
\end{enumerate}

\bigskip
 
\textbf{Partie C}

\medskip

%Dans cette partie, on étudie les coordonnées des images successives du sommet E du rectangle OEFG. On définit la suite des points $E_{n}\left(x_{n}~;~y_{n}\right)$ du plan par $E_{0} =$ E et la relation de récurrence :
%
%\[\begin{pmatrix}x_{n+1}\\y_{n+1}\end{pmatrix} = A\begin{pmatrix}x_{n}\\y_{n}\end{pmatrix},\] 
%
%où $\left(x_{n+1}~;~y_{n+1}\right)$ désignent les coordonnées du point $E_{n+1}$.
%
%Ainsi $x_{0} = 2$ et $y_{0} = 2$. 
%
%\medskip

\begin{enumerate}
\item %On admet que, pour tout entier $n \geqslant 1$, la matrice $A^n$ peut s'écrire sous la forme : $A^{n} = \begin{pmatrix}\alpha_{n}&\beta_{n}\\\beta_{n}&\alpha_{n}\end{pmatrix}$. 

%Démontrer par récurrence que, pour tout entier naturel $n \geqslant 1$, on a : 
%\[\alpha_{n} = 2^{n-1}  + \dfrac{1}{2^{n+1}} \quad \text{et}\quad  \beta_{n} = 2^{n-1}  - \dfrac{1}{2^{n+1}}.\] 
\emph{Initialisation} : pour $n = 1$, on a bien $A^1 = \begin{pmatrix}\frac{5}{4}&\frac{3}{4}\\\frac{3}{4}&\frac{5}{4} \end{pmatrix}$ et :

$\alpha_{1} = 2^0 + \frac{1}{2^2}$ et $\beta_{1} =  2^0 - \frac{1}{2^2}$.

\emph{Hérédité} : supposons que pour  tout  naturel $p$ tel que : $A^p = \begin{pmatrix}\alpha_{p}&\beta_{p}\\ \beta_{p}&\alpha_{p} \end{pmatrix}$ et 

\[\alpha_{p} = 2^{p-1}  + \dfrac{1}{2^{p+1}} \quad \text{et}\quad  \beta_{p} = 2^{p-1}  - \dfrac{1}{2^{p+1}}.\]

La relation $A^{p+1} = A \times A^p$ entraîne que :

$\alpha_{p+1} = \frac{5}{4}\alpha_{p} + \frac{3}{4}\beta_{p}$ et

$\beta_{p+1} = \frac{3}{4}\alpha_{p} + \frac{5}{4}\beta_{p}$, soit en utilisant la relation de récurrence :

$\alpha_{p+1} = \frac{5}{4}\left(2^{p-1}  + \dfrac{1}{2^{p+1}}\right) + \frac{3}{4}\left(2^{p-1}  - \dfrac{1}{2^{p+1}} \right) = \frac{8}{4}2^{p-1} + \frac{2}{4}\dfrac{1}{2^{p+1}} = 2^p + \dfrac{1}{2^{p+2}}$.

De même : 

$\beta_{p+1} = \frac{3}{4}\left(2^{p-1}  + \dfrac{1}{2^{p+1}}\right) + \frac{5}{4}\left(2^{p-1}  - \dfrac{1}{2^{p+1}} \right) = \dfrac{8}{4}2^{p-1} - \dfrac{2}{4}\dfrac{1}{2^{p+1}} = 2^p  - \dfrac{1}{2^{p+2}}$.

Donc les relations sont vraies au rang $p + 1$.

On a donc démontré par récurrence que pour tout entier naturel $n \geqslant 1$, on a : 

\[\alpha_{n} = 2^{n-1}  + \dfrac{1}{2^{n+1}} \quad \text{et}\quad  \beta_{n} = 2^{n-1}  - \dfrac{1}{2^{n+1}}.\]
\item 
	\begin{enumerate}
		\item %Démontrer que, pour tout entier naturel $n$, le point $E_{n}$ est situé sur la droite d'équation $y = x$.
		 
%On pourra utiliser que, pour tout entier naturel $n$, les coordonnées $\left(x_{n}~;~y_{n}\right)$ du point $E_{n}$ vérifient :

L'égalité 
\[\begin{pmatrix}x_{n}\\y_{n}\end{pmatrix} = A^n \begin{pmatrix}2\\2\end{pmatrix}.\]

se traduit par :

$\left\{\begin{array}{l c l}
x_{n}&=&2\alpha_{n} + 2\beta_{n}\\
y_{n}&=&2\beta_{n} + 2\alpha_{n}
\end{array}\right.$

On a quel que soit le naturel $n$, \, $x_{n} = y_{n}$.
 
		\item %Démontrer que la longueur O$E_{n}$ tend vers $+ \infty$ quand $n$ tend vers $+ \infty$.
OE$_{n}^2  = x_{n}^2 + y_{n}^2 = 2x_{n}^2$ ; 

Avec $x_{n} = 2\alpha_{n} + 2\beta_{n} = 2\left(\alpha_{n} + \beta_{n} \right)$ et $\alpha_{n} + \beta_{n} = 2^n$, on obtient 

OE$_{n}^2  = 2 \times 4 \left(2^n \right)^2 = 2^{2n + 3}$.

Or $\displaystyle\lim_{n \to + \infty}  2^{2n+3} = + \infty$, donc $\displaystyle\lim_{n \to + \infty} \text{OE}_{n} = + \infty$.
	\end{enumerate}
\end{enumerate}

\newpage

\begin{center} 
\textbf{Annexe}

\vspace{0,5cm}
 
\textbf{à rendre avec la copie}

\vspace{0,5cm}
 
\textbf{Exercice 2}

\vspace{0,5cm} 

\psset{unit=1.3cm}
\begin{pspicture*}(-5,-3.5)(5.1,5)
\psgrid[gridlabels=0pt,subgriddiv=1,gridwidth=0.2pt,gridcolor=orange]
\psaxes[linewidth=1pt](0,0)(-5,-3.5)(5,4.99)
\psaxes[linewidth=1.5pt]{->}(0,0)(1,1)
\psplot[plotpoints=8000,linewidth=1.25pt,linecolor=blue]{-4.5}{1.55}{2.71828 x exp}
\psplot[plotpoints=8000,linewidth=1.25pt,linecolor=red]{-1.5}{5}{1 2.71828 x neg exp sub}
\uput[dr](0,0){O}
\rput(1.05,3.5){\blue $\mathcal{C}_{f}$}
\rput(2.4,0.7){\red $\mathcal{C}_{g}$}
\psplot{-4}{4}{0.186374 x 1.68 add mul 0.186374 add}
\psplot{-4}{4}{2.15977 x 0.77 sub mul 2.15977 add}
\end{pspicture*}
\vspace{0,5cm}
\end{center}
\end{document}