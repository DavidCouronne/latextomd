\documentclass{cornouaille}
\dscornouaille
\begin{document}
\cornouaille{TS}{Eval 6}{24 janvier 2020}

\Bareme

\begin{exercice}[][8]
%STI2D Antilles septembre 2019
On note i le nombre complexe de module 1 et d'argument $\dfrac{\pi}{2}$.
\medskip

\begin{enumerate}
\item On considère le nombre complexe $z_1 = \sqrt{2}\text{e}^{\text{i}\frac{\pi}{4}}$.
	\begin{enumerate}
		\item Écrire $z_1$ sous forme algébrique.
		
\ldotcarreaux[2]
\setbar{1}
\begin{solution}
On sait que $\cos \frac{\pi}{4} = \sin \frac{\pi}{4} = \frac{\sqrt{2}}{2}$, donc 
		
$z_1 = \sqrt{2}\text{e}^{\text{i}\frac{\pi}{4}} = \sqrt{2}\left(\cos \frac{\pi}{4} + \text{i} \sin \frac{\pi}{4}\right) = \sqrt{2}\left(\frac{\sqrt{2}}{2} + \text{i}\frac{\sqrt{2}}{2} \right) = 1 + \text{i}$.
\end{solution}
		\item Vérifier que $z_1$ est solution de l'équation $(2 + \text{i})z = 1 + 3\text{i}$.

\ldotcarreaux[3]
\setbar{1}
\begin{solution}
$\bullet~~$En remplaçant dans l'équation :
		
$(2 + \text{i})(1 + \text{i}) = 1 + 3\text{i} \iff 2 + 2\text{i} + \text{i} - 1 = 1 + 3\text{i}$ qui est vraie.
		
$\bullet~~$En résolvant l'équation :

$(2 + \text{i})z = 1 + 3\text{i} \iff z = \dfrac{1 + 3\text{i}}{(2 + \text{i})} = \dfrac{(1 + 3\text{i})(2 - \text{i})}{(2 + \text{i})(2 - \text{i})} = \dfrac{2 + 3 + 6\text{i} - \text{i}}{4 + 1} = \dfrac{5 + 5\text{i}}{5} = 1 + \text{i}$	
\end{solution}
 	\end{enumerate}
\item Écrire le nombre complexe $z_2 = - 1 + \text{i}\sqrt{3}$ sous forme exponentielle.

\ldotcarreaux[5]
\setbar{1}
\begin{solution}
On calcule d'abord le module de $z_2$ :

$\left|z_2\right|^2 = 1 + 3 = 4 = 2^2$, donc $\left|z_2\right| = 2$.

On factorise ce module dans l'écriture de $z_2$ :

$z_2 = 2\left(- \dfrac{1}{2} + \text{i}\dfrac{\sqrt{3}}{2} \right)$.

On reconnait $- \dfrac{1}{2} = \cos \frac{2\pi}{3}$ et $\dfrac{\sqrt{3}}{2} = \sin \frac{2\pi}{3}$, donc 

$z_2 = 2\left(\cos \frac{2\pi}{3} + \text{i}\sin  \frac{2\pi}{3}\right) = 2\text{e}^{\text{i}\frac{2\pi}{3}}$.
\end{solution}
\item On considère $z_3$ le nombre complexe de module 4 et d'argument $\dfrac{7\pi}{6}$.

Vérifier  que $z_3 = z_1^2 \times z_2$.

\ldotcarreaux[5]
\setbar{1}
\begin{solution}
On a donc $z_3 = 4\text{e}^{\frac{7\pi}{6}}$.

D'autre part $z_1^2 \times z_2 = \left(\sqrt{2}\text{e}^{\text{i}\frac{\pi}{4}}\right)^2\times 2\text{e}^{\text{i}\frac{2\pi}{3}} = 2\text{e}^{\text{i}\frac{\pi}{2}}\times 2\text{e}^{\frac{2\pi}{3}} = 4 \text{e}^{\text{i}\left(\frac{\pi}{2} + \frac{2\pi}{3}\right)} = 4\text{e}{\text{i}\left(\frac{3\pi}{6}  + \frac{4\pi}{6}\right)} = 4\text{e}^{\text{i}\frac{7\pi}{6}}$.

On a donc $z_3 = z_1^2 \times z_2$.
\end{solution}
\item  Le plan complexe est muni d'un repère orthonormé \Ouv.
	
On considère les trois points A, B et C d'affixes respectives $z_{\text{A}} = 1 + \text{i}$,\; $z_{\text{B}} = - 1 + \text{i}\sqrt{3}$ et

$z_{\text{C}} = -2\sqrt{3} - 2\text{i}$.
	\begin{enumerate}
		\item Placer les points A, B et C dans le plan complexe ci-dessous. On laissera les éventuels traits de construction apparents.
		\begin{center}
\psset{unit=1cm}
\begin{pspicture}(-5.5,-5.5)(5.5,5.5)
\psaxes[linewidth=1.25pt,Dx=10,Dy=10](0,0)(-5.5,-5.5)(5.5,5.5)
\psaxes[linewidth=1.25pt]{->}(0,0)(1,1)
\pscircle[linewidth=1.2pt](0,0){1}
\pscircle(0,0){2}\pscircle(0,0){3}\pscircle(0,0){4}\pscircle(0,0){5}
\uput[dl](0,0){O} \uput[d](0.5,0){$\vect{u}$}\uput[l](0,0.5){$\vect{v}$}
\end{pspicture}
\end{center}

\setbar{2}
\begin{solution}
\begin{center}
\psset{unit=1cm}
\begin{pspicture}(-5.,-5.)(5.,5.)
\psgrid[gridlabels=0pt]
\psaxes[linewidth=1.25pt,Dx=10,Dy=10](0,0)(-5.5,-5.5)(5.5,5.5)
\psaxes[linewidth=1.25pt]{->}(0,0)(1,1)
\pscircle[linewidth=1.2pt](0,0){1}
\pscircle(0,0){2}\pscircle(0,0){3}\pscircle(0,0){4}\pscircle(0,0){5}
\uput[dl](0,0){O} \uput[d](0.5,0){$\vect{u}$}\uput[l](0,0.5){$\vect{v}$}
\psdots(1,1)(-1,1.732)(-3.464,-2)
\uput[ur](1,1){A}\uput[ul](-1,1.732){B} \uput[dl](-3.464,-2){C}
\pspolygon(0,0)(-1,1.732)(-3.464,-2)
\end{pspicture}
\end{center}
\end{solution}
		\item Démontrer que le triangle OBC est rectangle en O.

\ldotcarreaux[4]
\setbar{2}
\begin{solution}
On a $\vect{\text{OB}}\begin{pmatrix}-1\\\sqrt{3}\end{pmatrix}$ et $\vect{\text{OC}}\begin{pmatrix}-2\sqrt{3}\\- 2\end{pmatrix}$.

Calculons le produit scalaire $\vect{\text{OB}} \cdot \vect{\text{OC}} = - 1 \times (-2\sqrt{3}) + \sqrt{3} \times (- 2) = 2\sqrt{3} - 2\sqrt{3} = 0$ ; les vecteurs $\vect{\text{OB}}$ et $\vect{\text{OC}}$ sont orthogonaux, donc les droites (OB) et (OC) sont perpendiculaires, donc le triangle OBC est rectangle en O.
\end{solution}
	\end{enumerate}
\end{enumerate}

\end{exercice}

\begin{exercice}[][2]
%D'après TS métropole juin 2019 (le vrai-faux)
On note $u$ le nombre complexe $u=2+2i$ et on note $\overline{u}$ son conjugué.

Démontrer que: $u^{2020}+\overline{u}^{2020}=-2^{3031}$

\ldotcarreaux[5]
\setbar{2}
\begin{solution}
$|u|=...=2\sqrt{2}$ et après calculs on obtient: $\arg(u)=\frac{\pi}{4}$.\newline
En utilisant les propriétés sur les modules et arguments:\newline
$|u^{2020}|=|u|^{2020}=(2\sqrt{2})^{2020}=2^{2020}\times\sqrt{2}^{2020}=2^{2020}\times2^{1010}=2^{3030}$\newline
$\arg\left(u^{2020}\right)=2020\times\arg(u)=2020\times\frac{\pi}{4}=505\pi \equiv \pi [2\pi]$.\newline
Donc $u$ a pour module $2^{3030}$ et argument $\pi$. Donc $u=-2^{3030}$\newline
On a alors: $\overline{u}^{2020}=\overline{u^{2020}}=-2^{3030}$ et donc $u^{2020}+\bar{u}^{2020} = 2\times\left(-2^{3030}\right)= -2^{3031}$
\end{solution}

\end{exercice}
\end{document}