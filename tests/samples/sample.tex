\documentclass[10pt]{article}
\usepackage[T1]{fontenc}
\usepackage[utf8]{inputenc}
\usepackage{fourier}
\usepackage[scaled=0.875]{helvet}
\renewcommand{\ttdefault}{lmtt}
\usepackage{amsmath,amssymb,makeidx}
\usepackage[normalem]{ulem}
\usepackage{diagbox}
\usepackage{fancybox}
\usepackage{tabularx}
\usepackage{ulem}
\usepackage{multirow}
\usepackage{dcolumn}
\usepackage{textcomp}
%\usepackage[multiple]{footmisc}
%\usepackage[para]{manyfoot}
%\DeclareNewFootnote[para]{B}{alph}
\usepackage{lscape}
\usepackage{enumitem}
\newcommand{\euro}{\eurologo{}}
\usepackage{graphics,graphicx}
\usepackage{pstricks,pst-plot,pst-tree,pstricks-add}
\usepackage{tikz}
\usepackage[left=3.5cm, right=3.5cm, top=3cm, bottom=3cm]{geometry}
\newcommand{\R}{\mathbb{R}}
\newcommand{\N}{\mathbb{N}}
\newcommand{\D}{\mathbb{D}}
\newcommand{\Z}{\mathbb{Z}}
\newcommand{\Q}{\mathbb{Q}}
\newcommand{\C}{\mathbb{C}}
\renewcommand{\theenumi}{\textbf{\arabic{enumi}}}
\renewcommand{\labelenumi}{\textbf{\theenumi.}}
\renewcommand{\theenumii}{\textbf{\alph{enumii}}}
\renewcommand{\labelenumii}{\textbf{\theenumii.}}
\newcommand{\vect}[1]{\overrightarrow{\,\mathstrut#1\,}}
\def\Oij{$\left(\text{O}~;~\vect{\imath},~\vect{\jmath}\right)$}
\def\Oijk{$\left(\text{O}~;~\vect{\imath},~\vect{\jmath},~\vect{k}\right)$}
\def\Ouv{$\left(\text{O}~;~\vect{u}~;~\vect{v}\right)$}
\renewcommand{\i}{\text{i}}
\newcommand{\e}{\text{e}}
\usepackage{fancyhdr}
\usepackage[dvips]{hyperref}
%Tapuscrit : François Kriegk
\hypersetup{%
pdfauthor = {APMEP},
pdfsubject = {Terminale S},
pdftitle = {Amérique du Nord mai 2019},
allbordercolors = white,
pdfstartview=FitH}   
\usepackage[french]{babel}
\DecimalMathComma
\usepackage[np]{numprint}
\begin{document}
\setlength\parindent{0mm}
\rhead{\textbf{A. P{}. M. E. P{}.}}
\lhead{\small Terminale S}
\lfoot{\small{Amérique du Nord}}
\rfoot{\small{28 mai 2019}}
\pagestyle{fancy}
\thispagestyle{empty}
\begin{center}
    
{\Large \textbf{\decofourleft~Baccalauréat S Amérique du Nord 28 mai 2019~\decofourright}}
    
\bigskip
    
\textbf{Durée : 4 heures} \end{center}

\bigskip

\textbf{Exercice 1 \hfill 5 points} 

\textbf{Commun à tous les candidats} 

\medskip

\emph{Dans cet exercice et sauf mention contraire, les résultats seront arrondis à }$10^{-3}$.

\bigskip

Une usine fabrique des tubes.

\bigskip

\textbf{Partie A}

\bigskip

\emph{Les questions} \textbf{1.} \emph{et} \textbf{2.} \emph{sont indépendantes.}

\medskip

On s'intéresse à deux types de tubes, appelés tubes de type 1 et tubes de type 2.

\medskip

\begin{enumerate}
	\item Un tube de type 1 est accepté au contrôle si son épaisseur est comprise entre $1,35$ millimètre et $1,65$ millimètre.
	\begin{enumerate}
		\item On désigne par $X$ la variable aléatoire qui, à chaque tube de type 1 prélevé au hasard dans la production d'une journée, associe son épaisseur exprimée en millimètres. On suppose que la variable aléatoire $X$ suit la loi normale d'espérance $1,5$ et d'écart-type $0,07$.
		
\smallskip
		
On prélève au hasard un tube de type 1 dans la production de la journée. Calculer la probabilité que le tube soit accepté au contrôle.		
		\item L'entreprise désire améliorer la qualité de la production des tubes de type 1. Pour cela, on modifie le réglage des machines produisant ces tubes. On note $X_1$ la variable aléatoire qui, à chaque tube de type 1 prélevé dans la production issue de la machine modifiée, associe son épaisseur. On suppose que la variable aléatoire $X_1$ suit une loi normale d'espérance $1,5$ et d'écart-type $\sigma_1$.
		
\smallskip
		
Un tube de type 1 est prélevé au hasard dans la production issue de la machine modifiée. Déterminer une valeur approchée à $10^{-3}$ près de $\sigma_1$ pour que la probabilité que ce tube soit accepté au contrôle soit égale à 0,98. (On pourra utiliser la variable aléatoire $Z$ définie par $Z = \dfrac{X_1 - 1,5}{\sigma_1}$ qui suit la loi normale centrée réduite.)
	\end{enumerate}
\item Une machine produit des tubes de type 2. Un tube de type 2 est dit \og{}conforme pour la longueur \fg{} lorsque celle-ci, en millimètres, appartient à l’intervalle $[298~;~302]$. Le cahier des charges établit que, dans la production de tubes de type 2, une proportion de 2\,\% de tubes non \og{} conformes pour la longueur\fg{} est acceptable.
	
\smallskip
	
On souhaite décider si la machine de production doit être révisée. Pour cela, on prélève au hasard dans la production de tubes de type 2 un échantillon de $250$ tubes dans lequel 10 tubes se révèlent être non \og{}conformes pour la longueur\fg{}.	
	\begin{enumerate}
		\item Donner un intervalle de fluctuation asymptotique à $95$\,\% de la fréquence des tubes non \og{}conformes pour la longueur\fg{} dans un échantillon de $250$ tubes.		
		\item Décide-t-on de réviser la machine ? Justifier la réponse.
	\end{enumerate}
\end{enumerate}

\bigskip 

\textbf{Partie B}

\medskip

Des erreurs de réglage dans la chaine de production peuvent affecter l'épaisseur ou la longueur des tubes de type 2.

Une étude menée sur la production a permis de constater que :

\begin{itemize}
	\item 96\,\% des tubes de type 2 ont une épaisseur conforme;	
	\item parmi les tubes de type 2 qui ont une épaisseur conforme, 95\,\% ont une longueur conforme;	
	\item 3,6\,\% des tubes de type 2 ont une épaisseur non conforme et une longueur conforme.
\end{itemize}

On choisit un tube de type 2 au hasard dans la production et on considère les événements :

\begin{itemize}
	\item $E$ : \og{}l'épaisseur du tube est conforme\fg{};
	\item $L$ : \og{}la longueur du tube est conforme\fg{}.
\end{itemize}

On modélise l'expérience aléatoire par un arbre pondéré :

%:-+-+-+- Engendré par : http://math.et.info.free.fr/TikZ/Arbre/
\begin{center}
% Racine à Gauche, développement vers la droite
\begin{tikzpicture}[xscale=1,yscale=1]
% Styles (MODIFIABLES)
\tikzstyle{fleche}=[->,>=latex,thick]
\tikzstyle{noeud}=[fill=white,circle]
\tikzstyle{feuille}=[fill=white,circle]
\tikzstyle{etiquette}=[midway,fill=white,circle,inner sep=2pt]
% Dimensions (MODIFIABLES)
\def\DistanceInterNiveaux{3}
\def\DistanceInterFeuilles{1}
% Dimensions calculées (NON MODIFIABLES)
\def\NiveauA{(0)*\DistanceInterNiveaux}
\def\NiveauB{(1)*\DistanceInterNiveaux}
\def\NiveauC{(2)*\DistanceInterNiveaux}
\def\InterFeuilles{(-1)*\DistanceInterFeuilles}
% Noeuds (MODIFIABLES : Styles et Coefficients d'InterFeuilles)
\node[noeud] (R) at ({\NiveauA},{(1.5)*\InterFeuilles}) {};
\node[noeud] (Ra) at ({\NiveauB},{(0.5)*\InterFeuilles}) {$E$};
\node[feuille] (Raa) at ({\NiveauC},{(0)*\InterFeuilles}) {$L$};
\node[feuille] (Rab) at ({\NiveauC},{(1)*\InterFeuilles}) {$\overline{L}$};
\node[noeud] (Rb) at ({\NiveauB},{(2.5)*\InterFeuilles}) {$\overline{E}$};
\node[feuille] (Rba) at ({\NiveauC},{(2)*\InterFeuilles}) {$L$};
\node[feuille] (Rbb) at ({\NiveauC},{(3)*\InterFeuilles}) {$\overline{L}$};
% Arcs (MODIFIABLES : Styles)
\draw[fleche] (R.east) --(Ra) node[etiquette] {$\dots$};
\draw[fleche] (Ra.east) --(Raa) node[etiquette] {$\dots$};
\draw[fleche] (Ra.east) --(Rab) node[etiquette] {$\dots$};
\draw[fleche] (R.east) --(Rb) node[etiquette] {$\dots$};
\draw[fleche] (Rb.east) --(Rba) node[etiquette] {$\dots$};
\draw[fleche] (Rb.east) --(Rbb) node[etiquette] {$\dots$};
\end{tikzpicture}
\end{center}
%:-+-+-+-+- Fin

\begin{enumerate}
	\item Recopier et compléter entièrement cet arbre.	
	\item Montrer que la probabilité de l'événement $L$ est égale à 0,948.
\end{enumerate}

\vspace{5mm}

\textbf{Exercice 2 \hfill 4 points} 

\textbf{Commun à tous les candidats} 

\medskip

Le plan complexe est muni d'un repère orthonormé direct \Ouv. Dans ce qui suit, $z$ désigne un nombre complexe.

\medskip

Pour chacune des affirmations ci-dessous, indiquer sur la copie si elle est vraie ou si elle est fausse. Justifier. Toute réponse non justifiée ne rapporte aucun point.

\medskip

\begin{tabularx}{\linewidth}{l X}
\textbf{Affirmation 1 :}& L'équation $z - \text{i} = \text{i}(z + 1)$ a pour solution $\sqrt{2}\e^{\text{i}\frac{\pi}{4}}$.      \\ \\
	
\textbf{Affirmation 2 :}& Pour tout réel $x \in \left] -\dfrac{\pi}{2}~;~\dfrac{\pi}{2} \right[$, le nombre complexe $1 + \e^{2\text{i} x}$ admet pour forme exponentielle $2 \cos x \e^{-\text{i}x}$.\\ \\
	
\textbf{Affirmation 3 :}& Un point M d'affixe $z$ tel que $\big|z - \text{i}\big| = \big|z + 1\big|$ appartient à la droite d'équation $y = -x$.\\ \\
	
\textbf{Affirmation 4 :}& L'équation $z^5 + z - \text{i} + 1 = 0$ admet une solution réelle.
\end{tabularx}

\vspace{5mm}

\textbf{Exercice 3 \hfill 6 points} 

\textbf{Commun à tous les candidats} 

\medskip

\textbf{Partie A : établir une inégalité}
\medskip

Sur l'intervalle $[0~;~ +\infty[$, on définit la fonction $f$ par $f(x) = x - \ln(x + 1)$.
\begin{enumerate}
	\item Étudier le sens de variation de la fonction $f$ sur l'intervalle $[0~;~ +\infty[$.	
	\item En déduire que pour tout $x \in [0~;~ +\infty[, \quad \ln(x + 1) \leqslant x$.
\end{enumerate}

\bigskip

\textbf{Partie B : application à l'étude d'une suite}

On pose $u_0 = 1$ et pour tout entier naturel $n$, $u_{n+1} = u_n - \ln(1 + u_n)$. On admet que la suite de terme général $u_n$ est bien définie.

\begin{enumerate}
\item Calculer une valeur approchée à $10^{-3}$ près de $u_2$.	
\item 
		\begin{enumerate}
			\item Démontrer par récurrence que pour tout entier naturel $n, \quad u_n \geqslant 0$.		
			\item Démontrer que la suite $(u_n)$ est décroissante, et en déduire que pour tout entier naturel $n, \quad u_n \leqslant 1 $.			
			\item Montrer que la suite $(u_n)$ est convergente.
	\end{enumerate}
\item On note $\ell$ la limite de la suite $(u_n)$ et on admet que $\ell = f(\ell)$, où $f$ est la fonction définie dans la \textbf{partie A}. En déduire la valeur de $\ell$.
\item 
\begin{enumerate}
		\item Écrire un algorithme qui, pour un entier naturel $p$ donné, permet de déterminer le plus petit rang $N$ à partir duquel tous les termes de la suite $\left(u_n\right)$ sont inférieurs à $10^{-p}$.		
		\item Déterminer le plus petit entier naturel $n$ à partir duquel tous les termes de la suite $\left(u_n\right)$ sont inférieurs à $10^{-15}.\,$\footnote{La plupart des calculatrices et même des tableurs sont incapables de traiter cette question donnant même des résultats faux. Elle peut être sautée.}
	\end{enumerate}
\end{enumerate}

\vspace{5mm}

\textbf{Exercice 4 \hfill 5 points} 

\textbf{Candidats n’ayant pas suivi l’enseignement de spécialité}

On relie les centres de chaque face d'un cube ABCDEFGH pour former un solide IJKLMN comme sur la figure ci-dessous.

\begin{center}
	\begin{tikzpicture}[x={(-3:73mm)},y={(26:32mm)},z={(90:74mm)}]
	\coordinate (A) at (0,0,0);
	\coordinate (B) at (1,0,0);
	\coordinate (C) at (1,1,0);
	\coordinate (D) at (0,1,0);
	\coordinate (E) at (0,0,1);
	\coordinate (F) at (1,0,1);
	\coordinate (G) at (1,1,1);
	\coordinate (H) at (0,1,1);
	\coordinate (I) at (0.5,0.5,0);
	\coordinate (J) at (1,0.5,0.5);
	\coordinate (K) at (0.5,1,0.5);
	\coordinate (L) at (0,0.5,0.5);
	\coordinate (M) at (0.5,0,0.5);
	\coordinate (N) at (0.5,0.5,1);
	
	\draw (A) node [left]{A} -- (B) node [below right]{B} --
	(C) node [right]{C} -- (D) node [above left]{D}--cycle
	(E) node [left]{E} -- (F) node [below right]{F} --
	(G) node [right]{G} -- (H) node [above left]{H}--cycle
	(A)--(E) (B)--(F) (C)--(G) (D)--(H);
	
	\draw[line width=1pt] (I) node [below right] {I} --(J) node [right] {J}--(N)  node [above] {N}--(L)  node [left] {L}--cycle
	(L)--(M) node [below left] {M}--(J) (N)--(M)--(I);
	\draw[line width=1pt, dashed] (J)--(K) node [above right] {K} --(L) (N)--(K)--(I);

\end{tikzpicture}
\end{center}

Plus précisément, les points I, J, K, L, M et N sont les centres respectifs des faces carrées ABCD, BCGF, CDHG, ADHE, ABFE et EFGH (donc les milieux des diagonales de ces carrés).

\medskip
 
\begin{enumerate}
	\item Sans utiliser de repère (et donc de coordonnées) dans le raisonnement mené, justifier que les droites (IN) et (ML) sont orthogonales.
\end{enumerate}

\medskip

Dans la suite, on considère le repère orthonormé $\left(\text{A}~;~\vect{\text{AB}}~;~\vect{\text{AD}}~;~\vect{\text{AE}}\right) $ dans lequel, par exemple, le point N a pour coordonnées $\left(\dfrac{1}{2}~;~\dfrac{1}{2}~;~1\right) $.

\medskip

\begin{enumerate}[resume]
	\item 
		\begin{enumerate}
			\item Donner les coordonnées des vecteurs $ \vect{\text{NC}} $ et $ \vect{\text{ML}} $.			
			\item En déduire que les droites (NC) et (ML) sont orthogonales.			
			\item Déduire des questions précédentes une équation cartésienne du plan (NCI).
	\end{enumerate}	
\item 
\begin{enumerate}
		\item Montrer qu'une équation cartésienne du plan (NJM) est : $x - y + z = 1 $.		
		\item La droite (DF) est-elle perpendiculaire au plan (NJM)? Justifier.		
		\item Montrer que l'intersection des plans (NJM) et (NCI) est une droite dont on donnera un point et un vecteur directeur. 
Nommer la droite ainsi obtenue en utilisant deux points de la figure.
	\end{enumerate}
\end{enumerate}

\vspace{5mm}

\textbf{Exercice 4 \hfill 5 points} 

\textbf{Candidats ayant suivi l’enseignement de spécialité}

\medskip


Deux matrices colonnes $\begin{pmatrix} x\\y \end{pmatrix}$ et $\begin{pmatrix} x'\\y' \end{pmatrix}$ à coefficients entiers sont dites congrues modulo 5 si et seulement si $\left\{\begin{array}{l} x \equiv x'~[5]\\ y\equiv y'~[5]  \end{array} \right.$.


Deux matrices carrées d'ordre 2 $\begin{pmatrix} a&c\\b&d \end{pmatrix}$ et $\begin{pmatrix} a'&c'\\b'&d' \end{pmatrix}$ à coefficients entiers sont dites congrues modulo 5 si et seulement si $\left\{\begin{array}{l} a \equiv a'~[5]\\ b\equiv b'~[5] \\c \equiv c'~[5]\\ d\equiv d'~[5] \end{array} \right.$.

\bigskip

Alice et Bob veulent s'échanger des messages en utilisant la procédure décrite ci-dessous.

\parbox{0.53\linewidth}{
\begin{itemize}
\item Ils choisissent une matrice M carrée d'ordre 2, à coefficients entiers.
\item Leur message initial est écrit en lettres majuscules sans accent.
\item Chaque lettre de ce message est remplacée par une matrice colonne $\begin{pmatrix} x\\y \end{pmatrix}$ déduite du tableau ci-contre : $x$ est le chiffre situé en haut de la colonne et $y$ est le chiffre situé à la gauche de la ligne; par exemple, la lettre \textsf{T} d'un message initial correspond à la matrice colonne $\begin{pmatrix} 4\\3 \end{pmatrix}$.
\item On calcule une nouvelle matrice $\begin{pmatrix} x'\\y' \end{pmatrix}$ en multipliant $\begin{pmatrix} x\\y \end{pmatrix}$ à gauche par la matrice M : 
	
$\begin{pmatrix} x'\\y' \end{pmatrix} = \text{M} \begin{pmatrix} x\\y \end{pmatrix}$.	
	\item On calcule $r'$ et $t'$ les restes respectifs des divisions euclidiennes de $x'$ et $y'$ par 5. 
\end{itemize}} \hfill
\begin{tabular}{|*{6}{>{\centering \arraybackslash \rule[-2.5ex]{0pt}{6.5ex}}m{6mm}|}} \hline 
	&0&1&2&3&4 \\ \hline
0	&\textsf{A}&\textsf{B}&\textsf{C}&\textsf{D}&\textsf{E} \\ \hline
1	&\textsf{F}&\textsf{G}&\textsf{H}&\textsf{I}&\textsf{J} \\ \hline
2	&\textsf{K}&\textsf{L}&\textsf{M}&\textsf{N}&\textsf{O} \\ \hline
3	&\textsf{P}&\textsf{Q}&\textsf{R}&\textsf{S}&\textsf{T}\\ \hline
4	&\textsf{U}&\textsf{V}&\textsf{X}&\textsf{Y}&\textsf{Z} \\ \hline 
\multicolumn{6}{l}{~}\\
		
\multicolumn{6}{p{55mm}}{Remarque : la lettre \textsf{W} est remplacée par les deux lettres accolées \textsf{V}.}
	\end{tabular}
\begin{itemize}
		\item On utilise le tableau ci-contre pour obtenir la nouvelle lettre correspondant à la matrice colonne $\begin{pmatrix} r'\\t' \end{pmatrix}$.
\end{itemize}

\begin{enumerate}
	\item Bob et Alice choisissent la matrice $\text{M} = \begin{pmatrix} 1&2\\3&4 \end{pmatrix}$.	
	\begin{enumerate}
		\item Montrer que la lettre \og{}\textsf{T}\fg{} du message initial est codée par la lettre \og{}\textsf{U}\fg{} puis coder le message \og{}\textsf{TE}\fg{}.		
		\item On pose $\text{P} = \begin{pmatrix} 3&1\\4&2 \end{pmatrix}$. Montrer que les matrices PM et $\text{I} = \begin{pmatrix} 1&0\\0&1 \end{pmatrix}$ sont congrues modulo 5.		
		\item On considère A, A' deux matrices d'ordre 2 à coefficients entiers congrues modulo 5 et 
		
$\text{Z} =\begin{pmatrix} x\\y \end{pmatrix}$, $\text{Z}' =\begin{pmatrix} x'\\y' \end{pmatrix}$ deux matrices colonnes à coefficients entiers congrues modulo 5. Montrer alors que les matrices AZ et A'Z' sont congrues modulo 5.
	\end{enumerate}
\end{enumerate}

\bigskip

Dans ce qui suit on admet que si A, A' sont deux matrices carrées d'ordre 2 à coefficients entiers congrues modulo 5 et si B, B' sont deux matrices carrées d'ordre 2 à coefficients entiers congrues modulo 5 alors les matrices produit AB et A'B' sont congrues modulo 5.

\begin{enumerate}[start=2]
	\item[]	\begin{enumerate}[start=4]
		\item On note $\text{X}= \begin{pmatrix} x_1\\x_2 \end{pmatrix}$ et $\text{Y} = \begin{pmatrix} y_1\\y_2 \end{pmatrix}$ deux matrices colonnes à coefficients entiers. Déduire des questions précédentes que si MX et Y sont congrues modulo 5 alors les matrices X et PY sont congrues modulo 5; ce qui permet de \og{}décoder\fg{} une lettre chiffrée par la procédure utilisée par Alice et Bob avec la matrice M choisie.
		
		\item Décoder alors la lettre \og{}\textsf{D}\fg{}.
	\end{enumerate}
	\item On souhaite déterminer si la matrice $\text{R}=\begin{pmatrix} 1&2\\4&3	\end{pmatrix}$ peut être utilisée pour coder un message.
	\begin{enumerate}
		\item On pose $\text{S} = \begin{pmatrix} 2&2\\4&4	\end{pmatrix}$. Vérifier que la matrice RS et la matrice $\begin{pmatrix} 0&0\\0&0	\end{pmatrix}$ sont congrues modulo 5. 
		
		\item On admet qu'un message codé par la matrice R peut être décodé s‘il existe une matrice T telle que les matrices TR et I soient congrues modulo 5.
		Montrer que si c‘est le cas alors les matrices TRS et S sont congrues modulo 5 (par la procédure expliquée en question \textbf{1. d.} pour le codage avec la matrice M).
		
		\item En déduire qu‘un message codé par la matrice R ne peut être décodé.
	\end{enumerate}
\end{enumerate}
\end{document}