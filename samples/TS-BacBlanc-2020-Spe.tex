\documentclass[12pt]{cornouaille}
\dscornouaille


\begin{document}




\begin{exercice}[Pour les candidats ayant suivi l'enseignement de spécialité][5]
%Asie juin 2018

On s'intéresse à la figure suivante, dans laquelle $a$, $b$ et $c$ désignent les longueurs des hypoténuses des trois triangles rectangles en O dessinés ci-dessous.

\begin{center}
\psset{unit=3cm}
\begin{pspicture}(-0.2,-0.8)(3.4,1)
%\psgrid
\pspolygon(0,0)(1,0)(0,1)
\psline(1,0)(1.3,0)\psline(1.7,0)(2.1,0)\psline(2.1,0)(2.4,0)
\psline(2.7,0)(3.2,0)
\psline(0,1)(2.1,0)\psline(0,1)(3.2,0)
\psframe(0.15,0.15)
\psline[linestyle=dashed](1,0)(1,-0.2)\psline{<->}(0,-0.2)(1,-0.2)\uput[u](0.5,-0.2){1}
\psline[linestyle=dashed](2.1,0)(2.1,-0.4)\psline{<->}(0,-0.4)(2.1,-0.4)\uput[u](1.05,-0.4){$u$}
\psline[linestyle=dashed](3.2,0)(3.2,-0.6)\psline{<->}(0,-0.6)(3.2,-0.6)\uput[u](1.6,-0.6){$v$}
\uput[ur](0.5,0.5){$a$}\uput[ur](1.3,0.45){$b$}\uput[ur](1.85,0.5){$c$}
\psline[linestyle=dashed](0,0)(-0.2,0)\psline[linestyle=dashed](0,1)(-0.2,1)
\psline{<->}(-0.2,0)(-0.2,1)\uput[l](-0.2,0.5){1}
\psline[linestyle=dashed](1.2,0)(1.7,0)
\psline[linestyle=dashed](2.4,0)(2.7,0)
\psline[linestyle=dashed](0,0)(0,-0.6)
\end{pspicture}
\end{center}

\textbf{Problème :} on cherche les couples de \textbf{nombres entiers naturels non nuls} $(u,~v)$ tels que $ab = c$.

\medskip

\begin{enumerate}
\item Modélisation

Démontrer que les solutions du problème sont des solutions de l'équation :

\[(E) :\quad  v^2 - 2u^2 = 1\quad  (v \text{ et }\: u \: \text{ étant des entiers naturels non nuls}).\]

\begin{solution}
Dans le triangle rectangle, on applique le théorème de
Pythagore : $a^2=1^2+1^2=2$.
	
En appliquant le théorème dans les deux autres triangles rectangles on
peut écrire : 

$b^2=1^2+u^2=1+u^2$ et $c^2=1^2+v^2=1+v^2$.

$a$, $b$ et $c$ sont 3 longueurs donc sont des nombres positifs donc 
$ab=c \iff a^2b^2=c^2$; ainsi $2(1+u^2)=1+v^2$ c'est-à-dire
$2+2u^2=1+v^2$ donc $v^2-2u^2=1$.

Les solutions du problème sont donc les solutions $(u~;~v)$ de l'équation
$(E): \, v^2 - 2u^2=1$ où $u$ et $v $ sont des entiers naturels non nuls.
\end{solution}

\item  Recherche systématique de solutions de l'équation $(E)$

Recopier et compléter l'algorithme suivant pour qu'il affiche au cours de son exécution tous les couples solutions de l'équation pour lesquels $1 \leqslant u \leqslant \np{1000}$ et $1 \leqslant v \leqslant \np{1000}$.

\begin{center}
\begin{tabularx}{\linewidth}{|X|m{4.5cm}|}\hline
Pour $u$ allant de 1 à \ldots faire&Au cours de son exécution,\\
\hspace{0.5cm}Pour \ldots&l'algorithme affiche :\\
\hspace{1cm}Si \ldots&2 \quad 3\\
\hspace{1.5cm}Afficher $u$ et $v$&12 \quad 17\\
\hspace{1cm}Fin Si&70 \quad 99\\
\hspace{0.5cm}Fin Pour&408 \quad 577\\
Fin Pour&\\ \hline
\end{tabularx}
\end{center}

\begin{solution}
Algorithme :
	
\begin{center}
\begin{tabularx}{\linewidth}{|X|m{4.5cm}|}\hline
Pour $u$ allant de 1 à 1000 faire&Au cours de son exécution,\\
\hspace{0.5cm}Pour $v$ allant de 1 à 1000 &l'algorithme affiche :\\
\hspace{1cm}Si $v^2-2u^2=1$&2 \quad 3\\ \cdashline{2-2}
\hspace{1.5cm}Afficher $u$ et $v$&12 \quad 17\\ \cdashline{2-2}
\hspace{1cm}Fin Si&70 \quad 99\\ \cdashline{2-2}
\hspace{0.5cm}Fin Pour&408 \quad 577\\
Fin Pour&\\ \hline
\end{tabularx}
\end{center}
\end{solution}

\item Analyse des solutions éventuelles de l'équation $(E)$

On suppose que le couple $(u,~v)$ est une solution de l'équation $(E)$.
	\begin{enumerate}
		\item Établir que $u < v$.

\begin{solution}
Supposons que le couple $(u,v)$ est une solution de l'équation $(E)$ et que $u\geqslant v$. 
			
On a alors :  $u^2 \geqslant v^2$ et comme $2u^2 \geqslant u^2$, on a $2u^2 \geqslant v^2$ ce qui implique $v^2-2u^2\leqslant 0$. 

C'est impossible car $v^2-2u^2=1$.

Conclusion : $u < v$.	
\end{solution}
		\item  Démontrer que $n$ et $n^2$ ont la même parité pour tout entier naturel $n$.
		
\begin{solution}
On suppose que $n$ est pair. Il existe alors un entier naturel $k$
tel que $n=2k$.
			
On a alors $n^2=(2k)^2=4k^2=2(2k^2)$. Ainsi $n^2$ est pair.

On suppose que $n$ est impair. Il existe alors une entier naturel $q$ tel
que $n=2q+1$.

Alors $n^2=(q+1)2=4q^2+4q+1=2(2q^2+2q)+1$. Ainsi  $n^2$ est impair.


Conclusion : $n$ et $n^2$ ont la même parité.
\end{solution}

		\item  Démontrer que $v$ est un nombre impair.
		
\begin{solution}
Soit un couple solution $(u,v)$ du problème. On a  :
			
$v^2-2u^2=1 \iff v^2=2u^2+1$. 

Ainsi $v^2$ est impair. D'après la question précédente $v$ est aussi impair.
\end{solution}
		\item  Établir que $2u^2 =(v-1)(v+1)$.
		
En déduire que $u$ est un nombre pair.

\begin{solution}
On a : $v^2-2u^2=1 \iff 2u^2=v^2-1 \iff 2u^2=(v-1)(v+1)$.
			
Or $v$ est impair donc $v-1$ et  $v+1$ sont pairs : il existe ainsi un
entier naturel $k$ tel que $v-1=2k$ et $v+1=2(k+1)$.

Alors $2u^2=2k \times 2(k+1) \iff u^2=2k(k+1)$.

$u^2$ est donc pair. D'après la question \textbf{3.b.}, $u$ est donc pair.
\end{solution}
	\end{enumerate}
\item  Une famille de solutions
	
On assimile un couple de nombres entiers $(u,~v)$ à la matrice colonne $X = \begin{pmatrix}u\\v\end{pmatrix}$.
	
On définit également la matrice $A = \begin{pmatrix}3&2\\4&3\end{pmatrix}$.
	\begin{enumerate}
		\item Démontrer que si une matrice colonne $X$ est une solution de l'équation $(E)$, alors $AX$ est aussi une solution de l'équation $(E)$.
		
\begin{solution}
Soit $X=\begin{pmatrix}		u\\v		\end{pmatrix}$ une solution de l'équation $(E)$.
		
$AX=\begin{pmatrix} 3&2\\ 4&3 \end{pmatrix} \times \begin{pmatrix} u\\ v
\end{pmatrix}=\begin{pmatrix} 3u+2v\\ 4u+3v\end{pmatrix}$.

On a : $(4u+3v)^2-2(3u+2v)^2=16u^2+24uv+9v^2-18u^2-24uv-8v^2=v^2-2u^2=1$.

Ainsi $AX$ est aussi solution de $(E)$.
\end{solution}
		\item Démontrer que si une matrice colonne $X$ est une solution de l'équation $(E)$, alors pour tout entier naturel $n$,\: $A^n X$ est aussi une solution de l'équation $(E)$.
		
\begin{solution}
Montrons par récurrence que pour tout entier naturel $n$, $A^nX$ est une solution de l'équation $(E)$.
		
		\begin{itemize}
		\item Initialisation : $n=0$.
		
$A^0X=I_2X=X$. Et $X$ est solution de $(E)$. L'affirmation est donc vraie pour $n=0$.
		
		\item Soit $n \geqslant 0$ quelconque tel que $A^nX$ est une solution de $(E)$. 
		
$A^{n+1}X=A \times A^nX$ est une solution de $(E)$ d'après la question précédente.
		\item Conclusion : la propriété est vraie pour $n=0$ et est
héréditaire pour tout $n\geqslant 0$. D'après le raisonnement par récurrence, la propriété est vraie pour tout $n\geqslant 0$.
		\end{itemize}
		
On a donc démontré que, si $X$ est une solution de $(E)$, alors, pour tout entier naturel $n$, $A^nX$ est aussi une solution de $(E)$.
\end{solution}
		\item À l'aide de la calculatrice, donner un couple $(u,~v)$ solution de l'équation $(E)$ tel que
		
 $v > \np{10000}$.

\begin{solution}
On a : $3^2-2 \times 2^2=9-8=1$ donc le couple $(2;3)$ est solution de $(E)$.
		
$A^5 \times X = \begin{pmatrix} \np{13860}\\ \np{19601}\end{pmatrix}$. 
		
Donc le couple $(\np{13860}~;~\np{19601})$ est une solution de $(E)$ telle que $v > \np{10000}$.
\end{solution}
	\end{enumerate}
\end{enumerate}

	
\end{exercice}


\end{document}