\documentclass{cornouaille}
\begin{document}
\begin{exercice}[Pondichéry 2019]
\textbf{\textsc{Exercice 2} \hfill 6 points}
 
\textbf{Commun  à tous les candidats}

Le but de cet exercice est d'étudier la suite $\left(u_n\right)$ définie par la donnée de son premier terme $u_1$ et, pour tout entier naturel $n$ supérieur ou égal à 1, par la relation : \index{suite}

\[u_{n+1} = (n + 1)u_n - 1.\]

\smallskip
 
\textbf{Partie A}

\medskip 

\begin{enumerate}
\item  Vérifier, en détaillant le calcul, que si $u_1 = 0$ alors $u_4 = - 17$. 
\item Recopier et compléter l'algorithme ci-dessous pour qu'en saisissant préalablement dans $U$ une valeur de $u_1$ il calcule les termes de la suite $\left(u_n\right)$ de $u_2$ à $u_{13}$.\index{algorithme} 

\begin{center}
\begin{tabularx}{0.4\linewidth}{|X|}\hline
Pour $N$ allant de 1 à 12\\
\hspace{1cm} $U \gets $\\ 
Fin Pour \\ \hline
\end{tabularx}
\end{center}

\item On a exécuté cet algorithme pour $u_1 = 0,7$ puis pour $u_1 = 0,8$. 

Voici les valeurs obtenues. 

\begin{center}
\begin{tabularx}{0.8\linewidth}{|*{2}{X|}}\hline
\multicolumn{1}{|c|}{Pour $u_1 = 0,7$}&\multicolumn{1}{|c|}{Pour $u_1 = 0,8$}\\ \hline     
0,4					&0,6 \\  
0,2					&0,8\\   
$- 0,2$				&2,2 \\  
$- 2$				&10\\    
$- 13$				&59\\   
$- 92$				&412 \\  
$- 737$				&\np{3295}\\   
\np{-6634}			&\np{29654} \\ 
\np{-66341}			&\np{296539}\\   
\np{-729752}		&\np{3261928}\\   
\np{-8757025}		&\np{39143135} \\  
\np{-113841326}	&\np{508860754}\\ \hline
\end{tabularx}
\end{center}

Quelle semble être la limite de cette suite si $u_1 = 0,7$ ? Et si $u_1 = 0,8$ ? 
\end{enumerate}

\bigskip

\textbf{Partie B} 

\medskip

On considère la suite $\left(I_n\right)$ définie pour tout entier naturel $n$, supérieur ou égal à 1, par : 

\[I_n = \displaystyle\int_0^1 x^n \text{e}^{1 - x}\:\text{d}x.\]\index{suite d'intégrales}

On rappelle que le nombre e est la valeur de la fonction exponentielle en 1, c'est-à-dire que e $= \text{e}^1$.

\medskip 

\begin{enumerate}
\item Prouver que la fonction $F$ définie sur l'intervalle [0~;~1] par $F(x)=(- 1 - x)\text{e}^{1 - x}$ est une primitive sur l'intervalle [0~;~1] de la fonction $f$ définie sur l'intervalle [0~;~1] par $f(x) = x \text{e}^{1 - x}$.\index{primitive}
\item En déduire que $I_1 = \text{e} - 2$. 
\item On admet que, pour tout entier naturel $n$ supérieur ou égal à 1, on a : 

\[I_{n+1} =  (n+1)I_{n} - 1.\] 

Utiliser cette formule pour calculer $I_2$. 
\item 
	\begin{enumerate}
		\item Justifier que, pour tout nombre réel $x$ de l'intervalle [0~;~1] et pour tout entier naturel $n$ supérieur ou égal à 1, on a : $0 \leqslant x^n \text{e}^{1-x} \leqslant  x^n \text{e}$. 
		\item Justifier que : $\displaystyle\int_0^1 x^n\text{e}\:\text{d}x = \dfrac{\text{e}}{n+ 1}$. 
		\item En déduire que, pour tout entier naturel $n$ supérieur ou égal à 1, on a : $0 \leqslant I_n \leqslant \dfrac{\text{e}}{n+ 1}$. 
		\item Déterminer $\displaystyle\lim_{n \to + \infty} I_n$. 
	\end{enumerate}
\end{enumerate}

\bigskip

\textbf{Partie C}

\medskip 

Dans cette partie, on note $n$! le nombre défini, pour tout entier naturel n supérieur ou égal à 1, par : 1!=1 

2!$ =2 \times 1$ 

et si $n \geqslant 3$ : 
$n$! $= n \times  (n-1) \times  \ldots  \times 1$ 

On a ainsi par exemple 

3! $= 3\times 2\times 1 = 3\times(2 \times 1) = 3\times 2$! 

4! $= 4\times 3\times 2\times 1 = 4\times (3\times 2\times 1) = 4\times 3$! 

8! $= 8\times 7\times 6\times 5\times 4\times 3\times 2\times 1 = 8\times(7\times 6\times 5\times 4\times 3\times 2\times 1) = 8\times 7$! 

Et, plus généralement : 

\[(n+1)\text{!} = (n+1)  \times n\text{!}\]

\smallskip 

\begin{enumerate}
\item Démontrer par récurrence que, pour tout entier naturel $n$ supérieur ou égal à 1, on a :\index{récurrence} 

\[u_n =n\text{!}\left(u_1 - \text{e} + 2\right)+ I_n.\] 

On rappelle que, pour tout entier naturel $n$ supérieur ou égal à 1, on a :

\[ u_{n+1} = (n+1)u_{n} - 1\quad \text{ et } I_{n+1} = (n+1)I_{n} - 1.\] 

\item  On admet que : $\displaystyle\lim_{n \to + \infty} n\text{!} = + \infty$. 
	\begin{enumerate}
		\item Déterminer la limite de la suite $\left(u_n\right)$ lorsque $u_1 = 0,7$. 
		\item Déterminer la limite de la suite $\left(u_n\right)$ lorsque $u_1 = 0,8$.\index{limite de suite} 
	\end{enumerate}
\end{enumerate}
\end{exercice}

\begin{exercice}[Polynésie 2019]
\textbf{Exercice 3 \hfill  5 points}

\textbf{Commun à  tous les candidats}

\medskip

On considère la suite $\left(I_n\right)$ définie par $I_0 = \displaystyle\int_0^{\frac{1}{2}}\dfrac{1}{1 - x}\:\text{d}x$ et pour tout entier naturel $n$ non nul 

 \[I_n = \displaystyle\int_0^{\frac{1}{2}}\dfrac{x^n}{1 - x}\:\text{d}x.\]\index{suite d'intégrales}

\smallskip

\begin{enumerate}
\item Montrer que $I_0 = \ln (2)$. 
\item  
	\begin{enumerate}
		\item Calculer $I_0 - I_1$.
		\item En déduire $I_1$.  
	\end{enumerate}
\item 
	\begin{enumerate}
		\item Montrer que, pour tout entier naturel $n,\: I_n - I_{n+1} = \dfrac{\left(\frac{1}{2} \right)^{n+1}}{n+1}$. 
		\item Proposer un algorithme permettant de déterminer, pour un entier naturel $n$ donné, la valeur de $I_n$.\index{algorithme} 
	\end{enumerate}
\item Soit $n$ un entier naturel non nul. 

On admet que si $x$ appartient à l'intervalle $\left[0~;~\frac{1}{2}\right]$ alors $0 \leqslant \dfrac{x^n}{1 - x} \leqslant \dfrac{1}{2^{n-1}}$. 

	\begin{enumerate}
		\item Montrer que pour tout entier naturel $n$ non nul, $0 \leqslant  I_n \leqslant \dfrac{1}{2^n}$.
		\item En déduire la limite de la suite $\left(I_n\right)$ lorsque $n$ tend vers $+\infty$. 
	\end{enumerate}
\item Pour tout entier naturel $n$ non nul, on pose 

\[S_n = \dfrac{1}{2} + \dfrac{\left(\frac{1}{2} \right)^2}{2} + \dfrac{\left(\frac{1}{2} \right)^3}{3} + \ldots +\dfrac{\left( \frac{1}{2}\right)^n}{n}. \]

	\begin{enumerate}
		\item Montrer que pour tout entier naturel $n$ non nul, $S_n = I_0 - I_n$. 
		\item Déterminer la limite de $S_n$ lorsque $n$ tend vers $+\infty$. 
	\end{enumerate}
\end{enumerate}
\end{exercice}

\begin{exercice}[Liban 2019]
\textbf{Exercice 1 \hfill 5 points} 

\textbf{Commun à tous les candidats} 

\medskip

Le plan est muni d'un repère orthogonal (O, I, J).

\medskip 

\begin{enumerate}
\item On considère la fonction $f$ définie sur l'intervalle ]0~;~1] par 

\[f(x) = x(1 - \ln x)^2.\]\index{fonction logarithme népérien}

	\begin{enumerate}
		\item Déterminer une expression de la fonction dérivée de $f$ et vérifier que pour tout $x \in ]0~;~1]$, $f'(x) = (\ln x + 1)(\ln x - 1)$.\index{dérivée} 
		\item Étudier les variations de la fonction $f$ et dresser son tableau de variations sur l'intervalle ]0~;~1] (on admettra que la limite de la fonction $f$ en 0 est nulle). 
\end{enumerate}
	\end{enumerate}

On note $\Gamma$ la courbe représentative de la fonction $g$ définie sur l'intervalle ]0~;~1]  par $g(x) = \ln x$. 

Soit $a$ un réel de l'intervalle ]0~;~1]. On note $M_a$ le point de la courbe $\Gamma$ d'abscisse $a$ et $d_a$ la tangente à la courbe $\Gamma$ au point $M_a$. Cette droite $d_a$ coupe l'axe des abscisses au point $N_a$ et l'axe des ordonnées au point $P_a$ . 

On s'intéresse à l'aire du triangle O$N_aP_a$ quand le réel $a$ varie dans l'intervalle ]0~;~1]. 

\begin{enumerate}[resume]
\item Dans cette question, on étudie le cas particulier où $a = 0,2$ et on donne la figure ci-dessous. 

\begin{center}
\scalebox{0.8}{
\psset{xunit=10cm,yunit=2cm}
\begin{pspicture*}(-0.3,-3)(1.1,1.5)
\psgrid[unit=1cm,subgriddiv=1,gridlabels=0,gridcolor=lightgray](-1,-6)(11,3)
\psaxes[linewidth=1.25pt,labels=none]{->}(0,0)(-0.1,-3)(1.1,1.5)
\psplot[plotpoints=2000,linewidth=0.5pt]{-0.1}{0.8}{ x 5 mul 2.609 sub}
\psplot[plotpoints=2000,linewidth=1.25pt,linecolor=blue]{0.05}{1}{ x ln}
\psdots(0.2,-1.6)(0.5218,0)(1,0)(0,-2.609)(0,1)
\uput[dr](1,0){I}\uput[ur](0,1){J}\uput[dl](0,0){O}
\uput[dr](0.6,-0.5){\blue $\Gamma$}\uput[dr](0.7,0.9){$d_{0,2}$}
\uput[dr](0.2,-1.6){$M_{0,2}$}\uput[ul](0.5218,0){$N_{0,2}$}
\uput[ul](0,-2.609){$P_{0,2}$}
\end{pspicture*}
}%%% fin du scalebox
\end{center}    

\medskip

	\begin{enumerate}
		\item Déterminer graphiquement une estimation de l'aire du triangle O$N_{0,2}P_{0,2}$ en unités d'aire. 
		\item Déterminer une équation de la tangente $d_{0,2}$.\index{equation de la tangente@équation de la tangente} 
		\item Calculer la valeur exacte de l'aire du triangle O$N_{0,2}P_{0,2}$ . 

Dans ce qui suit, on admet que, pour tout réel $a$ de l'intervalle ]0~;~1], l'aire du triangle O$N_aP_a$ en unités d'aire est donnée par $\mathcal{A}(a) = \dfrac{1}{2}a (1 - \ln a)^2$.
	\end{enumerate}
\item  À l'aide des questions précédentes, déterminer pour quelle valeur de $a$ l'aire 
$\mathcal{A}(a)$ est maximale. Déterminer cette aire maximale. 
\end{enumerate}
\end{exercice}

\begin{exercice}[Antilles Guyanne 2019]
\textbf{\textsc{Exercice 1} \hfill 6 points}

\smallskip

\textsc{Commun à tous les candidats}

\bigskip

\textbf{Partie A}

\medskip

Soit $a$ et $b$ des nombres réels. On considère une fonction $f$ définie sur $[0~;~+\infty[$ par

\[f(x) = \dfrac{a}{1 + \text{e}^{-bx}}.\]

La courbe $\mathcal{C}_f$ représentant la fonction $f$ dans un repère orthogonal est donnée
ci-dessous.

La courbe $\mathcal{C}_f$ passe par le point A(0~;~0,5).
La tangente à la courbe $\mathcal{C}_f$ au point A passe par le point B(10~;~1).

\begin{center}
\psset{xunit=0.675cm,yunit=5cm,comma=true}
\begin{pspicture}(-1,-0.1)(20,1,2)
\psaxes[linewidth=1.25pt,Dy=0.1]{->}(0,0)(0,0)(20,1.2)
\psplot[plotpoints=2000,linewidth=1.25pt,linecolor=blue]{0}{20}{1 1 2.71828 x 0.2 mul neg exp add div}
\psplot[plotpoints=2000,linewidth=1.25pt]{0}{14}{x 0.05 mul 0.5 add}
\psdots(10,1)\uput[ul](10,1){B}
\uput[u](17,0.95){\blue $\mathcal{C}_f$}
\end{pspicture}
\end{center}

\medskip

\begin{enumerate}
\item Justifier que $a = 1$.

On obtient alors, pour tout réel $x \geqslant 0$,

\[f(x) = \dfrac{1}{1 + \text{e}^{-bx}}.\]

\item  On admet que la fonction $f$ est dérivable sur $[0~;~+\infty[$ et on note $f'$ sa fonction dérivée.

Vérifier que, pour tout réel $x \geqslant 0$\index{dérivée}

\[f'(x) = \dfrac{b\text{e}^{-bx}}{\left(1 + \text{e}^{-bx}\right)^2}.\]
\item  En utilisant les données de l'énoncé, déterminer $b$.\index{fonction exponentielle}
\end{enumerate}

\bigskip

\textbf{Partie B}

\medskip

La proportion d'individus qui possèdent un certain type d'équipement dans une
population est modélisée par la fonction $p$ définie sur $[0~;~+\infty[$ par

\[p(x) = \dfrac{1}{1 + \text{e}^{-0,2x}}.\]

Le réel $x$ représente le temps écoulé, en année, depuis le 1\up{er} janvier 2000.

Le nombre $p(x)$ modélise la proportion d'individus équipés après $x$ années.

Ainsi, pour ce modèle, $p(0)$ est la proportion d'individus équipés au 1\up{er} janvier 2000 et $p(3,5)$ est la proportion d'individus équipés au milieu de l'année 2003.

\medskip

\begin{enumerate}
\item Quelle est, pour ce modèle, la proportion d'individus équipés au 1er janvier
2010? On en donnera une valeur arrondie au centième.
\item 
	\begin{enumerate}
		\item Déterminer le sens de variation de la fonction $p$ sur $[0~;~+\infty[$.
		\item Calculer la limite de la fonction $p$ en $+\infty$.
		\item Interpréter cette limite dans le contexte de l'exercice.
 	\end{enumerate}
\item  On considère que, lorsque la proportion d'individus équipés dépasse 95\,\%, le
marché est saturé.
	
Déterminer, en expliquant la démarche, l'année au cours de laquelle cela se
produit.
\item  On définit la proportion moyenne d'individus équipés entre 2008 et 2010 par

\[m = \dfrac{1}{2}\displaystyle\int_8^{10} p(x)\:\text{d}x.\]
	
	\begin{enumerate}
		\item Vérifier que, pour tout réel $x \geqslant 0$,

\[p(x) = \dfrac{\text{e}^{0,2x}}{1 + \text{e}^{0,2x}}.\]

		\item En déduire une primitive de la fonction $p$ sur $[0~;~+\infty[$.\index{primitive}
		\item Déterminer la valeur exacte de $m$ et son arrondi au centième.
	\end{enumerate}
\end{enumerate}
\end{exercice}

\begin{exercice}[Métropole juin 2019]
\textbf{\Large Exercice 1 \hfill  6 points}

\textbf{Commun à  tous les candidats}

{\large\bf Partie A}

\medskip

% \def\e{\text{e}}
On considère la fonction $f$ définie sur l'ensemble $\R$ des nombres réels par :

\[f(x)=\dfrac{7}{2}-\dfrac{1}{2}\left(\e^x+\e^{-x}\right)\]\index{fonction exponentielle}

\begin{enumerate}
\item
	\begin{enumerate}
		\item Déterminer la limite de la fonction $f$ en $+\infty$.
		\item Montrer que la fonction $f$ est strictement décroissante sur l'intervalle $[0~;~+\infty[$.\index{dérivée}
		\item Montrer que l'équation $f(x) = 0$ admet, sur l'intervalle $[0~;~+\infty[$, une unique solution, que l'on note $\alpha$.
	\end{enumerate}
\item En remarquant que, pour tout réel $x$, $f(-x) = f(x)$, justifier que l'équation $f(x)=0$ admet exactement deux solutions dans $\R$ et qu'elles sont opposées.
\end{enumerate}

{\large\bf Partie B}

\medskip

Les \textbf{serres en forme de tunnel} sont fréquemment utilisées pour la culture des plantes fragiles; elles limitent les effets des intempéries ou des variations de température.

Elles sont construites à partir de plusieurs arceaux métalliques identiques qui sont ancrés au sol et supportent une bâche en plastique.
\medskip

Le plan est rapporté à un repère orthonormé d'unité 1 mètre. La fonction $f$ et le réel $\alpha$ sont définis dans la \textbf{partie A}. Dans la suite de l'exercice, on modélise un arceau de serre par la courbe $\mathcal{C}$ de la fonction $f$ sur l'intervalle $[-\alpha~;~+\alpha]$.

On a représenté ci-dessous la courbe $\mathcal{C}$ sur l'intervalle  $[-\alpha~;~+\alpha]$.


\begin{center}
\psset{unit=1.5cm,linewidth=1.2pt}
\def\xmin{-3} \def\xmax{3.9} \def\ymin{-.85} \def\ymax{3.5}
\begin{pspicture*}(-3,-.85)(3.9,3.5)
%\psframe[fillstyle=solid,fillcolor=lightgray](-1,0)(1,2.2)
\psaxes[linewidth=1.25pt](0,0)(-3.5,-.5)(3,3.5)
\psplot{-1.92}{1.92}{3.5 2.7181828 x -1 mul exp 2.7181828 x exp add .5 mul sub}
\psline(-3.2,2.2)(-3.2,-3.5)
\psline(-4.8,2.2)(-4.8,-3.5)
\psframe(-5.4,-3.5)(-2.6,-5)
\psline[linestyle=dashed](-3,2.5)(3,2.5)
\psline[arrows=<->,linewidth=1.pt](2.5,0.)(2.5,2.5)
\rput[l](2.65,1.25){\textbf{hauteur}}
\rput[l](-1.65,1.5){$\mathcal{C}$}
\end{pspicture*}
\end{center}

On admettra que la courbe $\mathcal{C}$ admet l'axe des ordonnées pour axe de symétrie.

\begin{enumerate}
	\item Calculer la hauteur d'un arceau.
	\item \begin{enumerate}
		\item Dans cette question, on se propose de calculer la valeur exacte de la longueur de la courbe~$\mathcal{C}$ sur l'intervalle $[0~;~\alpha]$. On admet que cette longueur est donnée, en mètre, par l'intégrale : 
		
\[I=\displaystyle\int_0^{\alpha}\sqrt{1+(f'(x))^2}\ dx\]

		Montrer que, pour tout réel $x$, on a : $1+(f'(x))^2=\dfrac{1}{4}(\e^x+\e^{-x})^2$
		\item En déduire la valeur de l'intégrale $I$ en fonction de $\alpha$.
		
Justifier que la longueur d'un arceau, en mètre, est égale à : $\e^{\alpha}-\e^{-\alpha}$.
	\end{enumerate}
\end{enumerate}

{\large\bf Partie C}

\medskip

On souhaite construire une serre de jardin en forme de tunnel.

On fixe au sol quatre arceaux métalliques, dont la forme est celle décrite dans la partie précédente, espacés de 1,5~mètre, comme indiqué sur le schéma ci-dessous.

Sur la façade sud, on prévoit une ouverture modélisée sur le schéma par le rectangle $ABCD$ de largeur 1~mètre et de longueur 2~mètres.

\begin{center}
	\psset{unit=1.2cm,linewidth=1.2pt}
	\def\xmin{-3} \def\xmax{7} \def\ymin{-.85} \def\ymax{6}
	\begin{pspicture*}(-3,-.85)(7,6)
	\psframe[fillstyle=vlines,fillcolor=lightgray](-.5,0)(.5,2)
	%\psaxes[linewidth=1.25pt](0,0)(-3.5,-.5)(3,3.5)
	\psplot{-1.92}{1.92}{3.5 2.7181828 x  -1 mul exp 2.7181828 x exp add .5 mul sub}
	\psplot{-.92}{2.92}{3.5 2.7181828 x 1 sub  -1 mul exp 2.7181828 x 1 sub exp add .5 mul sub 1 add}
	\psplot{.08}{3.92}{3.5 2.7181828 x 2 sub  -1 mul exp 2.7181828 x 2 sub exp add .5 mul sub 2 add}
	\psplot{1.08}{4.92}{3.5 2.7181828 x 3 sub  -1 mul exp 2.7181828 x 3 sub exp add .5 mul sub 3 add}
\psline(-3.2,2.2)(-3.2,-3.5)
\psline(-4.8,2.2)(-4.8,-3.5)
\psframe(-5.4,-3.5)(-2.6,-5)
\psline[linestyle=dashed](-1.92,0)(1.08,3)(4.92,3)
\psline(-1.92,0)(1.92,0)(4.92,3)
\psline(-.96,2)(0.96,2)
\psline(0,2.5)(3,5.5)
\psline[arrows=<->,linewidth=1.pt](2.5,0.)(3.5,1)
\rput[l](3.2,.5){\textbf{1m50}}
\uput[35](.5,0){$C$}
\uput[-35](.5,2){$B$}
\uput[135](-.5,0){$D$}
\uput[-35](-.5,2){$A$}
\rput[b](-.5,-0.5){\textbf{Façade sud}}
\rput[b](3.5,3){\textbf{Façade nord}}
\end{pspicture*}
\end{center}

On souhaite connaître la quantité, exprimée en m$^2$, de bâche plastique nécessaire pour réaliser cette serre.

Cette bâche est constituée de trois parties, l'une recouvrant la façade nord, l'autre la façade sud (sauf l'ouverture), la troisième partie de forme rectangulaire recouvrant le toit de la serre.

\begin{enumerate}
	\item Montrer que la quantité de bâche nécéssaire pour recouvrir les façades sud et nord est donnée, en m$^2$, par : \[\mathcal{A}=4\int_0^{\alpha}f(x)\ dx-2\]
	\item On prend $1,92$ pour valeur approchée de $\alpha$. Déterminer, au m$^2$ près, l'aire totale de la bâche plastique nécessaire pour réaliser cette serre.
\end{enumerate}
\end{exercice}

\begin{exercice}[Antilles Guyane spetembre 2019]
\textbf{\textsc{Exercice 3} \hfill 5 points} 

\textbf{Commun à tous les candidats}

\medskip

Soit $g$ la fonction définie sur $]0~;~+\infty[$ par 

\[g(x) = 4x - x \ln x.\]\index{fonction logarithme népérien}

On admet que la fonction $g$ est dérivable sur $]0~;~ +\infty[$ et on note $g'$ sa dérivée.

\bigskip

\textbf{Partie A}

\medskip

\begin{minipage}{0.48\linewidth}Le graphique ci-contre représente une partie de la 
courbe représentative de la fonction $g$ obtenue par un élève sur sa calculatrice. Cet élève émet les deux conjectures suivantes :

\setlength\parindent{10mm}
\begin{itemize}
\item[$\bullet~~$] il semble que la fonction $g$ soit positive ;
\item[$\bullet~~$] il semble que la fonction $g$ soit strictement
croissante.
\end{itemize}
\setlength\parindent{0mm}
\end{minipage}\hfill 
\begin{minipage}{0.48\linewidth}
\psset{xunit=0.5cm,yunit=0.3cm,labelFontSize=\scriptstyle}
\begin{pspicture}(-1,-1)(10.5,10.5)
\psaxes(0,0)(-0.9,-0.9)(10.5,10.5)
\psplot[plotpoints=2000,linewidth=1.25pt,Dy=2]{0.001}{4}{4 x mul x x ln mul sub}
\end{pspicture}
\end{minipage}

\medskip

L'objectif de cette partie est de valider ou d'invalider chacune de ces conjectures.

\medskip

\begin{enumerate}
\item Résoudre l'équation $g(x) = 0$ sur l'intervalle $]0~;~ +\infty[$.
\item Déterminer le signe de $g(x)$ sur l'intervalle $]0~;~ +\infty[$.
\item Les conjectures de l'élève sont-elles vérifiées ?

\end{enumerate}

\bigskip

\textbf{Partie B}

\medskip

Dans cette partie, on poursuit l'étude de la fonction $g$.

\medskip

\begin{enumerate}
\item 
	\begin{enumerate}
		\item On rappelle que
		
\[\displaystyle\lim_{t \to + \infty} \dfrac{\ln t}{t} = 0.  \]

En déduire que

\[\displaystyle\lim_{x \to 0} x \ln x = 0.\]

		\item Calculer la limite de $g(x)$ lorsque $x$ tend vers $0$.
	\end{enumerate}
\item
	\begin{enumerate}
		\item Démontrer que, pour tout réel $x$ strictement positif, $g'(x) = 3 - \ln x$.\index{dérivée}
		\item Dresser le tableau de variations de la fonction $g$.
	\end{enumerate}
\item On désigne par $G$ la fonction définie sur $]0~;~ +\infty[$ par 

\[G(x) = \dfrac{1}{4}x^2(9 - 2\ln x).\]

On admet que la fonction $G$ est dérivable sur $]0~;~ +\infty[$.
	\begin{enumerate}
		\item Démontrer que la fonction $G$ est une primitive de la fonction $g$ sur $]0~;~ +\infty[$.\index{primitive}
		\item L'affirmation suivante est-elle vraie ?
		
\og Il n'existe aucun réel $\alpha$ strictement supérieur à 1 tel que $\displaystyle\int_1^{\alpha} g(x) \:\text{d}x = 0$. \fg
	\end{enumerate}
\end{enumerate}
\end{exercice}
\end{document}