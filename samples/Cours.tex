\documentclass{cornouaille}


\begin{document}
\fexo{Terminale spécialité}{Chapitre 4 : Limites de fonctions et continuité}{Cours}

\section{Introduction : observations}

\subsection{Quelles sont les positions possibles d'une courbe en $+\infty$ ?}

\begin{tabularx}{\linewidth}{|c|*{2}{>{\centering \arraybackslash}X|}} \hline
\includegraphics[scale=1.2]{figintro1.png} & Plus $x$ devient grand, plus $f(x)$ devient grand. $\displaystyle \lim_{x\rightarrow +\infty} f(x)=+\infty$ \\ \hline
\includegraphics[scale=0.5]{figintro2.png} &    Plus $x$ devient grand, plus $f(x)$ devient petit. $\displaystyle \lim_{x\rightarrow +\infty} f(x)=-\infty$\\ \hline
\includegraphics[scale=1.4]{figintro3.png} & Plus $x$ devient grand, $f(x)$ devient de plus en plus proche de 1 sans jamais l'atteindre. $\displaystyle \lim_{x\rightarrow +\infty} f(x)=1$ \\ \hline
\includegraphics[scale=1.2]{figintro4.png} &  Plus $x$ grandit, plus $f(x)$ ne cesse de varier. $\displaystyle \lim_{x\rightarrow +\infty} f(x)$ n'existe pas.\\ \hline 
\includegraphics[scale=1.4]{figintro5.png} & Plus $x$ devient grand, plus $f(x)$ devient grand. $\displaystyle \lim_{x\rightarrow +\infty} f(x)=+\infty$ \\ \hline
\end{tabularx}

\medskip

\underline{Remarque} : On peut visualiser les mêmes types de limite en $-\infty$.

\subsection{Quelles sont les positions possibles d'une courbe en une valeur finie $a$ ?}

\begin{tabularx}{\linewidth}{|c|*{2}{>{\centering \arraybackslash}X|}} \hline
\includegraphics[scale=0.4]{figintro6.png}  $f$ est définie en $a$ & Plus $x$ devient proche de $a$, plus $f(x)$ devient proche de $f(a)$. $\displaystyle \lim_{x\rightarrow a} f(x)=f(a)$ \\ \hline
\includegraphics[scale=1.3]{figintro7.png} $f$ est définie en $a$ &    Plus $x$ devient proche de $a$, plus l'ordonnée se rapproche de deux valeurs différentes, suivant que $x$ se rapproche par la gauche ou par la droite. $\displaystyle \lim_{x\rightarrow a} f(x)=$ n'existe pas.\\ \hline
\includegraphics[scale=0.4]{figintro8.png} $f$ n'est pas définie en $a$ & Plus $x$ se rapproche de $a$, plus $f(x)$ devient proche de $b$. $\displaystyle \lim_{x\rightarrow a} f(x)=b$ \\ \hline
\includegraphics[scale=0.4]{figintro9.png} $f$ n'est pas définie en $a$ &  Plus $x$ se rapproche de $a$ par la droite, plus $f(x)$ devient grand. $\displaystyle \lim_{x\rightarrow a^{+}} f(x) = +\infty$.\\ \hline 
\includegraphics[scale=0.4]{figintro10.png} $f$ n'est pas définie en $a$ &  Plus $x$ se rapproche de $a$ par la gauche, plus $f(x)$ devient petit. $\displaystyle \lim_{x\rightarrow a^{-}} f(x) =  -\infty$ \\ \hline
\end{tabularx}


\section{Limite d'une fonction en l'infini}

Dans toute cette partie, $\mathcal{C}_f$ désigne la courbe représentative de la fonction $f$ dans un repère quelconque du plan.

\subsection{Limite infinie en l'infini}

\begin{definition}
La fonction $f$ a  pour limite $+\infty$ en $+\infty$ si tout intervalle de $\mathbb{R}$  du type  $]a~;~+\infty[$
contient toutes les valeurs de $f(x)$ pour $x$ assez grand. On note alors : $ \lim_{x\to+\infty}f(x)=+\infty$.\\
\end{definition}

\begin{exemple}
Soit $f$ la fonction racine carrée. 
En effet, $\sqrt{x}$ devient aussi grand que l'on veut à mesure que $x$ augmente.\\
On a $\lim_{x\to+\infty}\sqrt{x}=+\infty$.\\
Soit un intervalle ouvert $I=]a~;~+\infty[$. Alors, $f(x)$ sera toujours dans $I$ pour $x$ assez grand.\\
 Graphiquement, si on considère le demi-plan supérieur de frontière une droite d'équation \mbox{$y=a$}, il existe toujours une valeur de $a$ au delà de laquelle $\mathcal{C}_f$ ne sort plus de ce demi-plan.
\begin{center}
\psset{xunit=0.5cm,yunit=1cm,algebraic=true}
\begin{pspicture*}(-2.5,-0.6)(20.5,4.8)
\psframe*[linecolor=H4](0,3.75)(20.5,4.8)
\psgrid[yunit=0.5cm,subgriddiv=1,linewidth=0.5pt,gridcolor=A3,gridlabels=0pt](0,0)(21,10)
\psaxes[linewidth=0.8pt,Dx=1,Dy=1,ticksize=-2pt]{->}(0,0)(-0.5,-0.25)(20.5,4.8)
\uput[r](1.2,2.7){\textcolor{B2}{$\boldsymbol{\mathcal{C}_{f}:y=\sqrt{x}}$}}
\psplot[linecolor=B2,plotpoints=5000,linewidth=1pt]{0.01}{20.5}{sqrt(x)}
\psline[linewidth=0.8pt,linestyle=dashed,linecolor=B2](0,3.75)(20.5,3.75)
\uput[l](0,3.75){\textcolor{H1}{$a$}}
\end{pspicture*}
\end{center}
\end{exemple}

\begin{remarque}
\begin{itemize}
\item On définit de façon analogue :
$\lim_{x\to +\infty}f(x)=-\infty$,
$\lim_{x\to -\infty}f(x)=+\infty$ et
$\lim_{x\to -\infty}f(x)=-\infty$.
\item Il existe des fonctions qui n'admettent pas de limite en
  l'infini. Par exemple, les fonctions sinus et cosinus n'admettent
  de limite ni en $+\infty$, ni en $-\infty$.
\item Une fonction qui tend vers $+\infty$ lorsque $x$ tend vers $+\infty$ n'est pas forcément croissante(voir observations figure 5)
\end{itemize}
\end{remarque}




%
%\begin{methode} [Étudier les limites d'une fonction en l'infini \label{A2-methode1}]
%
%  

%\textbf{Exercice:}


%
%  \textbf{Conjecturer avec une calculatrice}
%
%  $g$ est la fonction sur
%  $]-\infty\ ;\ 0\left[\cup \right]0\ ;\ +\infty [$ par
%  $g = \frac{3x-7}{x}$.
%
%  On obtient avec la calculatrice le tableau de valeurs de la
%  fonction $g$ ci-contre.
%  \begin{enumerate}
%  \item Conjecturer la limite de $g$ en $+\infty{}$.
%  \item Démontrer cette conjecture à l'aide de la définition.
%  \item Interpréter graphiquement cette conjecture.
%  \end{enumerate}
%
%  \begin{center}
%%    \includegraphics{}
%  \end{center}
%
%  

%\textbf{Correction}


%
%  \begin{enumerate}
%  \item On examine la limite de $g$ en $+\infty{}$ et on remarque que
%    la limite est 3.
%
%  \item On a, pour tout réel $x$ non nul $[$ par $(x)=3- \frac{7}{x}$.\par
%    $g(x) \in{}]3-\varepsilon\ ;\ 3+\varepsilon [\ \Longleftrightarrow
%    (3-\varepsilon )<3-\ \frac{7}{x}<\left(3+\varepsilon
%    \right)\Longleftrightarrow \ (-\varepsilon )<3-\
%    \frac{7}{x}<\left(+\varepsilon \right)$.\par
%    De plus, pour $x ]0\ ;\ +\infty[$, donc
%    $\varepsilon > \frac{7}{x}>0 \Longleftrightarrow
%    \frac{1}{\varepsilon} < \frac{x}{7} $,
%    (car la fonction inverse est strictement croissante sur
%    $]0\ ;\ +\infty[$), $\Longleftrightarrow \frac{7}{\varepsilon}<x$.\par
%    Donc, l'intervalle $]3-\varepsilon\ ;\ 3+\varepsilon[$, est centré
%    en 3, et contient toutes les valeurs de $g (x)$ pour
%    $x> \frac{7}{\varepsilon}$, c'est-à-dire pour $x$ assez grand $\lim_{x\rightarrow+\infty{}}$
%    $g(x) = 3$.
%  \item La courbe représentant $g$ dans n'importe quel repère admet
%    une asymptote horizontale d'équation $(y = 3)$.
%  \end{enumerate}
%\end{methode}
\subsection{Limite finie en l'infini}

\begin{definition}
Soit $f$ une fonction définie au moins sur un intervalle de $\mathbb{R}$ du type $]a~;~+\infty[$.\\
La fonction $f$ a pour limite $\ell$ en $+\infty$ si tout intervalle ouvert contenant $\ell$
contient toutes les valeurs de $f(x)$ pour $x$ assez grand. On note alors : $ \lim_{x\to+\infty}f(x)=\ell$.\\
\end{definition}

\begin{exemple}
Soit $f$ la fonction définie sur $]0~;~+\infty[$ par $f(x)=\dfrac{1}{x}+1$. \\
L'inverse de $x$ se rapproche de $0$ à mesure que $x$ augmente.
On a donc $\lim_{x\to+\infty}\left(\dfrac{1}{x}+1\right)=1$.\\
\\
Soit un intervalle ouvert $I$ tel que $1\in I$. Alors, $f(x)$ sera toujours dans $I$ pour $x$ assez grand. Graphiquement, aussi étroite que soit une bande parallèle à la droite d'équation $y=1$ et qui la contient, il existe toujours une valeur de $x$ au delà de laquelle $\mathcal{C}_f$ ne sort plus de cette bande.
\begin{center}
\psset{xunit=0.5cm,yunit=2cm,algebraic=true}
\begin{pspicture*}(-2.5,-0.3)(20.5,2.4)
\psframe*[linecolor=H4](0,0.875)(20,1.125)
\psgrid[yunit=0.5cm,subgriddiv=1,linewidth=0.5pt,gridcolor=A3,gridlabels=0pt](0,0)(20,9)
\psaxes[linewidth=0.8pt,Dx=1,Dy=1,ticksize=-2pt]{->}(0,0)(-0.5,-0.125)(20.5,2.4)
\uput[r](1,2.125){\textcolor{B2}{$\boldsymbol{\mathcal{C}_{f}:y=\dfrac{1}{x}+1}$}}
\psplot[linecolor=B2,plotpoints=5000,linewidth=1pt]{0.8}{20}{1/x+1}
\psline[linewidth=0.8pt,linestyle=dashed,linecolor=B2](0,1.125)(20,1.125)
\psline[linewidth=0.8pt,linestyle=dashed,linecolor=B2](0,0.875)(20,0.875)
\psline[linewidth=0.8pt,linecolor=H1](0,1)(20,1)
\uput[l](0,0.875){\textcolor{H1}{$1-\varepsilon$}}
\uput[l](0,1.125){\textcolor{H1}{$1+\varepsilon$}}
\end{pspicture*}
\end{center}
\end{exemple}

\begin{definition}[Asymptote horizontale]
La droite d'équation $y=\ell$ est \textbf{asymptote horizontale} à $\boldsymbol{\mathcal{C}_f}$ \textbf{en} $\boldsymbol{+\infty}$   si $\lim_{x\to+\infty} f(x)=\ell$.
\end{definition}

  \begin{remarque}
  On définit de façon analogue $\lim_{x\to-\infty} f(x)=\ell$
  qui caractérise une asymptote horizontale à $\mathcal{C}_f$ en $-\infty$ d'équation $y=\ell$.
  \end{remarque}

  \begin{exemple}
On a vu précédemment que $\lim_{x\to+\infty}\left(\dfrac{1}{x}+1\right)=1$. On a aussi $\lim_{x\to-\infty}\left(\dfrac{1}{x}+1\right)=1$.\\
Donc, la droite d'équation $y=1$  est asymptote horizontale à la courbe $\mathcal{C}_f$  en $+\infty$ et en $-\infty$ .
  \end{exemple}

\subsection{Limites des fonctions de référence en $\pm\infty$}

\begin{propriete}[Limites finies des fonctions usuelles en $\pm\infty$]
Soit $n$ un entier naturel non nul.
\begin{colitemize}{2}
\item $ \lim_{x\to+\infty}\sqrt{x}=\lim_{x\to+\infty}x^n=+\infty$
\item $ \lim_{x\to-\infty}x^n=\left\{\begin{array}{@{}ll}
+\infty &  \text{pour~} n \text{~pair} \\
-\infty &  \text{pour~} n \text{~impair}
\end{array}\right.$
\item $ \lim_{x\to+\infty}\frac{1}{\sqrt{x}}=\lim_{x\to+\infty}\frac{1}{x^n}=0$
\item $ \lim_{x\to-\infty}\frac{1}{x^n}=0$
\item $ \lim_{x\to-\infty} e^x=0$
\item $ \lim_{x\to+\infty} e^x=+\infty$
\end{colitemize}
\end{propriete}




\section{Limite infinie en un réel}

%\subsection{Limite finie en un réel}
%
%\begin{definition} [Limite finie en un réel \emph{a}]
%  Dire qu'une fonction $f$ a pour limite le nombre $\ell $ en $a$,
%  signifie que tout intervalle ouvert de centre $\ell$, même
%  extrêmement petit, contient tous les nombres $f(x)$ dès lors que $x$
%  est \textbf{"suffisamment proche de} $a$" (c'est-à-dire dès que $x$
%  appartient à un intervalle du type $]a - \alpha~; a + \alpha[$ et à
%  $D_{f}$).
%
%  On écrit~alors $\mathop{\lim }\limits_{x\to a} f(x)=\ell $.
%\end{definition}
%
%Les distances de $f(x)$ à $\ell$ tendent donc vers 0. Or la distance
%de $f(x)$ à $\ell$ s'écrit $|f(x) - \ell|$.
%
%Donc dire que $\mathop{\lim }\limits_{x\to a} f(x)=\ell $ signifie que
%$\mathop{\lim } \limits_{x\to a} \left|f(x)-\ell \right|=0$.
%
%\subsection{Limite infinie en un réel}

\begin{definition}
Soit $f$ une fonction définie sur un intervalle ouvert de $\mathbb{R}$ du type $]x_0-\varepsilon~;~x_0[$ ou $]x_0~;~x_0+\varepsilon[$.\\
La fonction $f$ a pour limite $+\infty$ en $x_0$ si tout intervalle de $\mathbb{R}$ du type 
$]A~;~+\infty[$  contient toutes les valeurs de $f(x)$ pour $x$ assez proche de $x_0$. On note alors : $ \lim_{x\to x_0}f(x)=+\infty$.\\
\end{definition}


\begin{exemple}
Soit $f$ la fonction définie sur $]1~;~+\infty[$ par $f(x)=\dfrac{1}{x-1}$.\\
Si $x$ tend  $1$, alors $x-1$ tend vers $0$ et son inverse tend vers $+\infty$.On a $\lim_{x\to 1}\dfrac{1}{x-1}=+\infty$.\\

Soit un intervalle ouvert $I=]1~;~1+\varepsilon[$. Alors, $f(x)$ sera toujours dans $I$ pour $x$ assez proche de $x_0$. Graphiquement,  $\mathcal{C}_f$ peut être aussi proche que l'on veut de la droite d'équation $x=1$.
\begin{center}
\psset{xunit=2cm,yunit=0.2cm,algebraic=true}
\begin{pspicture*}(-.3,-2.5)(4.5,21)
\psframe*[linecolor=H4](1,0)(1.2,21)
\psgrid[yunit=1cm,subgriddiv=1,linewidth=0.5pt,gridcolor=A3,gridlabels=0pt](0,0)(5,10)
\psaxes[linewidth=0.8pt,Dx=1,Dy=5,ticksize=-2pt]{->}(0,0)(-0.25,-2)(4.5,21)
\uput[r](2,2.5){\textcolor{B2}{$\boldsymbol{\mathcal{C}_{f}:y=\dfrac{1}{x-1}}$}}
\psplot[linecolor=B2,plotpoints=5000,linewidth=1pt]{1.01}{4.5}{1/(x-1)}
\psline[linewidth=0.8pt,linestyle=dashed,linecolor=B2](1.2,0)(1.2,21)
\psline[linewidth=0.8pt,linecolor=H1](1,0)(1,21)
\uput{7pt}[d](1.3,0){\textcolor{H1}{$1+\varepsilon$}}
\end{pspicture*}
\end{center}
\end{exemple}


\begin{definition}[Asymptote verticale]
  La droite d'équation $x=x_0$ est \textbf{asymptote
    verticale} à $\boldsymbol{\mathcal{C}_f}$ si
  $\lim_{x\to x_0} f(x)=\pm\infty$.
\end{definition}

\begin{exemple}
On a vu précédemment que $\lim_{x\to 1}\dfrac{1}{x-1}=+\infty$.\\
Donc, la droite d'équation $x=1$  est asymptote verticale à l'hyperbole $\mathcal{C}_f$.
\end{exemple}


\begin{remarque}
\begin{itemize}
\item Lorsque $x$ tend vers $x_0$, cela peut parfois se faire en augmentant ou en diminuant. On parle alors de
limite de $f$ à gauche (resp. droite) en $x_0$ qu'on note $ \lim_{\substack{x\to x_0\\ x<x_0}}f(x)$
(resp. $ \lim_{\substack{x\to x_0\\ x>x_0}}f(x)$).
\item Une fonction admet une limite en $x_0$ si, et seulement si, $f$ admet des
 limites à droite et à gauche en $x_0$ qui sont égales (ce qui n'est pas toujours le cas).
\item  Une fonction peut très bien ne pas avoir de limite du tout en un point.\par
Par exemple, la fonction $x \mapsto \sin \dfrac{1}{x}$ n'a pas de limite en 0.
\end{itemize}
\end{remarque}


\begin{propriete}[Limites finies des fonctions usuelles en $\boldsymbol{0}$]
Soit $n$ un entier naturel non nul.
\begin{colitemize}{2}
\item $\lim_{\substack{x\to 0\\ x>0}}\frac{1}{\sqrt{x}}=\lim_{\substack{x\to 0\\ x>0}}\frac{1}{x^n}=+\infty$
\item $\lim_{\substack{x\to 0\\ x<0}}\dfrac{1}{x^n}=\left\{\begin{array}{@{}ll}
+\infty &  \text{pour~} n \text{~pair} \\
-\infty &  \text{pour~} n \text{~impair}
\end{array}\right.$
\end{colitemize}
\end{propriete}


\begin{remarque}
L'aperçu de la courbe représentative d'une fonction avec une  calculatrice ou un logiciel peut aider à conjecturer une limite (et donc éventuellement une asymptote  à la courbe).\\

Soit $f$ une fonction dont on a un aperçu du graphe $\mathcal{C}$. Déterminer son ensemble de définition $\mathcal{D}$, puis conjecturer les limites aux bornes de $\mathcal{D}$ et les  asymptotes à $\mathcal{C}$.
\begin{colenumerate}{2}
\item $f:x\mapsto \dfrac{x^3-1}{x^3+1}$\par
% \fcolorbox{fondTI}{fondTI}{
  %\includegraphics[scale=.75]{meth_graph1.eps}
\item $f:x\mapsto 2x-\sqrt{4x^2-1}$ \par
% \fcolorbox{fondTI}{fondTI}{
  %\includegraphics[scale=.75]{meth_graph2.eps}
\end{colenumerate}

\end{remarque}  

\textbf{Correction}

  \begin{enumerate}
  \item  $\mathcal{D}=\mathbb{R}\setminus\{-1\}$. A priori, on aurait : $\lim_{x\to\pm+\infty}f(x)=1$ ; $\lim_{\substack{x\to -1\\ x<-1}}f(x)=+\infty$ et $\lim_{\substack{x\to -1\\ x>-1}}f(x)=-\infty$.\par
        $\mathcal{C}$ aurait alors une asymptote  horizontale d'équation $y=1$ en $\pm\infty$ et une asymptote verticale d'équation $x=-1$.
  \item  $\mathcal{D}=]-\infty~;~-\tfrac{1}{2}[\,\cup\,]\tfrac{1}{2}~;~+\infty[$. On a : $\lim_{x\to -1/2}f(x)=-1$ et $\lim_{x\to 1/2}f(x)=1$ et, il semblerait que $\lim_{x\to-\infty}f(x)=-\infty$ et $\lim_{x\to +\infty}f(x)=0$.\par
        $\mathcal{C}$ aurait alors une asymptote horizontale d'équation $y=0$ (l'axe des abscisses) en $+\infty$.\par
         
  \end{enumerate}
  

\section{Opérations sur les limites}

\enlargethispage{5mm}

\begin{propriete}[Limite d'une somme, d'un produit et d'un quotient de deux fonctions]
\begin{colitemize}{3}
\item Limite d'une somme :\par
\begin{tabular}{@{}|c|c|c|}
\hline
\rowcolor{white} $f$ & $g$ & $f+g$ \\
\hline
$\ell$ & $\ell'$ & $\ell+\ell'$ \\
\hline
$\ell$ & $\infty$ &  $\infty$ \\
\hline
$+\infty$ & $+\infty$  & $+\infty$  \\
\hline
$-\infty$ & $-\infty$  & $-\infty$  \\
\hline
 \rowcolor{Noir!10}  $+\infty$ &  $-\infty$ &  $???$ \\
 \hline
 \end{tabular}
\item Limite d'un produit :\par
\begin{tabular}{@{}|c|c|c|}
\hline
\rowcolor{white} $f$ & $g$ & $fg$ \\
\hline
$\ell$ & $\ell'$ & $\ell\ell'$ \\
\hline
$\ell\!\neq\!0$ & $\infty$ &  $\infty$ \\
\hline
$\infty$ & $\infty$  & $\infty$  \\
\hline
 \rowcolor{Noir!10}  $0$ &  $\infty$ &  $???$ \\
 \hline
 \end{tabular}
 \item Limite d'un quotient :\par
\begin{tabular}{@{}|c|c|c|}
\hline
\rowcolor{white} $f$ & $g$ & $f/g$ \\
\hline
$\ell$ & $\ell'\!\neq\!0$ & $\ell/\ell'$ \\
\hline
$\ell\!\neq\!0$ & $0$ &  $\infty$ \\
\hline
$\ell$ & $\infty$  & $0$  \\
\hline
\rowcolor{Noir!10}   $0$ & $0$  & $???$  \\
\hline
 \rowcolor{Noir!10}  $\infty$ &  $\infty$ &  $???$ \\
 \hline
 \end{tabular}
\end{colitemize}
\end{propriete}



\begin{remarque}
\begin{itemize}
\item $\infty$ peut signifier $+\infty$ ou $-\infty$. Les règles du signe d'un produit ou d'un quotient demeurent.
\item Pour la limite de la différence $f-g$, on considère la limite de la somme $f+(-g)$.
\item Les quatre lignes grises des tableaux correspondent aux quatre cas d'indétermination :\\
\begin{minipage}{1.1\linewidth}
\begin{colitemize}{4}
\item " $\boldsymbol{(+\infty)+(-\infty)}$ "
\item " $\boldsymbol{0\times\infty}$ "
\item " $\boldsymbol{\dfrac{0}{0}}$ "
\item " $\boldsymbol{\dfrac{\infty}{\infty}}$ "
\end{colitemize}
\end{minipage}

\end{itemize}
\end{remarque}

\begin{exemple}
Soit $f:x\mapsto  \frac{x-1}{x^2-4}$.
Calculons les limites de $f(x)$ aux bornes de son domaine de définition.
\end{exemple}


\section{Limite d'une fonction composée}

\subsection{\textbf{Fonction composée}}

Une composée de deux fonctions correspond à un enchaînement de deux fonctions l'une après l'autre.\\
Par exemple, composons la fonction $f:x\mapsto1-x$ suivie de $g:x\mapsto\sqrt{x}$. On peut ainsi schématiser :
\begin{center}\renewcommand{\arraystretch}{0.7}
\begin{tabular}{c@{}c@{}c@{}c@{}c}
$x$ & $\,\mapsto\,$ & $1-x$ & $\,\mapsto\,$ & $\sqrt{1-x}$. \\
 & $f$ & &  $g$ &
\end{tabular}\renewcommand{\arraystretch}{1.}
\end{center}
Cependant, on voit que la fonction $g$ ne peut s'appliquer que si l'ensemble des images par la fonction $f$ est inclus dans l'ensemble de définition de $g$.\\
Ainsi, pour appliquer ici la racine carrée, il faut que $1-x\geqslant0$ c'est-à-dire que $x\leqslant1$.\\
La composée existe donc dans le schéma suivant où on précise les ensembles de départ et d'arrivée pour $f$ :
\begin{center}\renewcommand{\arraystretch}{0.7}
\begin{tabular}{c@{}c@{}c@{}c@{}c}
$]-\infty~;~2]$ & $\rightarrow$ & $[0~;~+\infty[$ &  $\rightarrow$ & $\mathbb{R}$ \\
$x$ & $\,\mapsto\,$ & $1-x$ & $\,\mapsto\,$ & $\sqrt{1-x}$ \\
 & $f$ & &  $g$ & \\
\end{tabular}\renewcommand{\arraystretch}{1.}
\end{center}
En composant $f$ suivie de $g$, on a ainsi défini sur $]-\infty~;~1]$ la fonction $x\mapsto\sqrt{1-x}$.\\


\begin{definition}
Soit $f$ une fonction définie sur $E$ et à valeurs dans  $F$, et soit $g$ une fonction définie sur $F$.\\
\textbf{La composée de} $\boldsymbol{f}$ \textbf{suivie de} $\boldsymbol{g}$ est la fonction notée $\boldsymbol{g\circ f}$ définie sur $E$ par $g\circ f(x)=g(f(x))$.
\end{definition}

\begin{remarque}
Il ne faut pas confondre $g\circ f$ et $f\circ g$ qui sont, en général, différentes.
\end{remarque}

\begin{exemple}
 En reprenant $f$ et $g$ de l'exemple précédent, définissons $f\circ g$.\\
La composée de $g$ suivie de $f$ est possible en partant de l'ensemble de définition de $g$ :
\begin{center}\renewcommand{\arraystretch}{0.7}
\begin{tabular}{c@{}c@{}c@{}c@{}c}
$[0~;~+\infty[$ & $\rightarrow$ & $[0~;~+\infty[$ &  $\rightarrow$ & $\mathbb{R}$ \\
$x$ & $\,\mapsto\,$ & $\sqrt{x}$ & $\,\mapsto\,$ & $1-\sqrt{x}$ \\
 & $g$ & &  $f$ & \\
\end{tabular}\renewcommand{\arraystretch}{1.}
\end{center}
En composant $g$ suivie de $f$, on a ainsi défini sur $[0~;~+\infty[$ la fonction $x\mapsto1-\sqrt{x}$.
\end{exemple}


\subsection{Théorème de composition des limites}

\begin{theoreme}
Soit $h$ la composée de la fonction $f$ suivie de $g$ et
$\textcolor{B2}{\alpha}$, $\textcolor{J1}{\beta}$ et
$\textcolor{H2!60!black}{\gamma}$ trois réels ou $\pm\ \infty$.\\
Si $\lim_{x\to\textcolor{B2}{\alpha}} f(x)=\textcolor{J1}{\beta}$ et $\lim_{x\to\textcolor{J1}{\beta}} g(x)=\textcolor{H2!60!black}{\gamma}$, alors $\lim_{x\to\textcolor{B2}{\alpha}} h(x)=\textcolor{H2!60!black}{\gamma}$.
\end{theoreme}

\begin{exemple}
Déterminons la limite en $-\infty$ de la fonction  $g\circ f$ de l'exemple précédent.\\
La composée de $f:x\mapsto 1-x$ suivie de $g:x\mapsto\sqrt{x}$ est $h:x\mapsto \sqrt{1-x}$ définie sur $]-\infty~;~1]$.\\
Or, $\lim_{x\to\textcolor{B2}{-\infty}} (1-x)=\textcolor{J1}{+\infty}$ (par somme) et $\lim_{x\to\textcolor{J1}{+\infty}}  \sqrt{x}=\textcolor{H2!60!black}{+\infty}$ (limite de référence).\\
Donc, d'après le théorème de composition, $\lim_{x\to\textcolor{B2}{-\infty}} \sqrt{1-x}=\textcolor{H2!60!black}{+\infty}$.
\end{exemple}



\begin{methode}[Déterminer une limite de fonction ]
On applique les propriétés d'opérations sur les limites.\\
Si la limite est indéterminée, " $+\infty+(-\infty)$ ", " $0\times\infty$ ", " $\dfrac{\infty}{\infty}$ " ou " $\dfrac{0}{0}$ ", on essaye de :
\begin{itemize}
\item factoriser par le terme prépondérant ;
\item multiplier par la quantité conjuguée si des racines carrées interviennent ;
\item effectuer un changement de variable (voir théorème de
  composition des limites).
\end{itemize}
D'autres techniques existent et seront vues ultérieurement.
  

\textbf{Exercice:}

 \label{meth_limite_fonction}
Calculer les limites suivantes :
\begin{colenumerate}{3}
\item $\lim_{x\to+\infty}\left(\sqrt{x+1}-\sqrt{x}\right)$
\item $\lim_{x\to+\infty}\dfrac{2x^2-3x+1}{x^2-1}$
\item $\lim_{x\to4}\dfrac{x - 4}{\sqrt{x} - 2}$
\end{colenumerate}

  

\textbf{Correction}


Ces  limites sont indéterminées (respectivement formes " $\infty-\infty$ ", " $\dfrac{\infty}{\infty}$ " et " $\dfrac{0}{0}$ ").
  \begin{enumerate}
  \item On multiplie le numérateur et le dénominateur par la quantité
    conjuguée de $\sqrt{x+1}-\sqrt{x}$ :\par
    $\sqrt{x+1}-\sqrt{x}=\dfrac{\left(\sqrt{x+1}-\sqrt{x}\right)\left(\sqrt{x+1}+\sqrt{x}\right)}{\sqrt{x+1}+\sqrt{x}}=\dfrac{1}{\sqrt{x+1}+\sqrt{x}}$.\par
    Or, par composition :
    $\lim_{x\to\textcolor{B2}{+\infty}}(x+1)=\textcolor{J1}{+\infty}$
    et
    $\lim_{x\to\textcolor{J1}{+\infty}}\sqrt{x}=\textcolor{H1}{+\infty}$
    donc
    $\lim_{x\to\textcolor{B2}{+\infty}}\sqrt{x+1}=\textcolor{J1}{+\infty}$.\par
    Et, par somme :
    $\lim_{x\to+\infty}\left(\sqrt{x+1}+\sqrt{x}\right)=+\infty$. Donc,
    par inverse :
    $\lim_{x\to+\infty}\dfrac{1}{\sqrt{x+1}+\sqrt{x}}=0$.



\item Divisons le numérateur et le dénominateur par $x^2$. Alors,
  $\dfrac{2x^2-3x+1}{x^2-1}=
  \dfrac{2-\tfrac{3}{x}+\tfrac{1}{x^2}}{1-\tfrac{1}{x^2}}$.\par
  Or, par somme :
  $\lim_{x\to+\infty}\left(2-\dfrac{3}{x}+\dfrac{1}{x^2}\right)=2$
  et
  $\lim_{x\to+\infty}\left(1-\dfrac{1}{x^2}\right)=1$.



Donc, par quotient : $\lim_{x\to+\infty}\dfrac{2-\tfrac{3}{x}+\tfrac{1}{x^2}}{1-\tfrac{1}{x^2}}=2$.
  \item Changeons de variable en posant $u=\sqrt x$. Si $x$ tend vers $4$, alors $u$ tend vers $2$.\par
  $\dfrac{x-4}{\sqrt x-2}=\dfrac{u^2-4}{u-2}=\dfrac{(u+2)(u-2)}{u-2}=u+2$ pour $u\neq2$. Donc, par somme : $\lim_{u\to2}(u+2)=4$.
  \end{enumerate}
\end{methode}

\section{Limites et comparaison}

\subsection{Théorème de comparaison}

\begin{theoreme}
Soit $f$ et $g$ deux fonctions telles que $f(x)\leqslant g(x)$ sur un intervalle $]\alpha~;~+\infty[$ de $\mathbb{R}$.
\begin{colitemize}{2}
\item $\lim_{x\to+\infty} f(x)=+\infty\Rightarrow\lim_{x\to+\infty} g(x)=+\infty$.
\item $\lim_{x\to+\infty} g(x)=-\infty\Rightarrow\lim_{x\to+\infty} f(x)=-\infty$.
\end{colitemize}



Soit $f$ et $g$ deux fonctions telles que $f(x)\leqslant g(x)$ sur un intervalle $]-\infty~;~\beta[$ de $\mathbb{R}$.

\begin{colitemize}{2}
\item $\lim_{x\to-\infty} f(x)=+\infty\Rightarrow\lim_{x\to-\infty} g(x)=+\infty$.
\item $\lim_{x\to-\infty} g(x)=-\infty\Rightarrow\lim_{x\to-\infty} f(x)=-\infty$.
\end{colitemize}



Soit $f$ et $g$ deux fonctions telles que $f(x)\leqslant g(x)$ sur un intervalle $]\alpha~;~\beta[$ de $\mathbb{R}$  et $x_0\in]\alpha~;~\beta[$.

\begin{colitemize}{2}
\item $\lim_{x\to x_0} f(x)=+\infty\Rightarrow\lim_{x\to x_0} g(x)=+\infty$.
\item $\lim_{x\to x_0} g(x)=-\infty\Rightarrow\lim_{x\to x_0} f(x)=-\infty$.
\end{colitemize}
\end{theoreme}


\begin{exemple}
Déterminons la limite en $+\infty$ et en $-\infty$ de $f(x)=x+\sin x$.\\
La limite de $\sin x$ en $\pm \infty$ est indéterminée donc, celle de $f(x)$ aussi.\\
Mais pour tout $x\in\mathbb{R}$, $-1\leqslant\sin x\leqslant 1$ donc $x-1\leqslant x+\sin x\leqslant x+1$. Ainsi :
\begin{itemize}
\item De $x-1\leqslant x+\sin x$ ~~et~~ $\lim_{x\to +\infty} (x-1)=+\infty$, on déduit que $\lim_{x\to +\infty} (x+\sin x)=+\infty$.
\item De $x+\sin x\leqslant x+1$ ~~et~~ $\lim_{x\to -\infty} (x+1)=-\infty$, on déduit que $\lim_{x\to -\infty} (x+\sin x)=-\infty$.
\end{itemize}


\end{exemple}

\subsection{Conséquence : croissance comparée}
\begin{theoreme}
Pour tout entier naturel $n$, $\lim_{n\rightarrow +\infty} \frac{e^x}{x^n}=+\infty$.
\end{theoreme}

\underline{Démonstration} :\\

Pour tout réel $y$, $y<e^y$.\\
On pose : $y=\frac{x}{n+1}$. L'inégalité devient: $\frac{x}{n+1}<\e^{\frac{x}{n+1}}$.\\
La fonction $t \mapsto t^{n+1}$ est strictement croissante sur $[0 ; +\infty[$.\\
Donc : $\left(\frac{x}{n+1}\right)^{n+1}<\left(\e^{\frac{x}{n+1}}\right)^{n+1} \iff \frac{x^{n+1}}{(n+1)^{n+1}}<e^x \iff \frac{x}{(n+1)^{n+1}}<\frac{e^x}{x^n}$.\\
Or, $\lim_{x\rightarrow+\infty} \frac{x}{(n+1)^{n+1}}=+\infty$, donc d'après le théorème de comparaison, $\lim_{x\rightarrow+\infty} \frac{e^x}{x^n}=+\infty$.



\subsection{Théorème d'encadrement dit " des gendarmes "}

\begin{theoreme}
Soit deux réels $\alpha$ et $\ell$ et trois fonctions $f$, $g$ et $h$ telles que, pour $x>\alpha$, on a $f(x)\leqslant g(x)\leqslant h(x)$.\\
Si  $\lim_{x\rightarrow+\infty}f(x)=\lim_{x\rightarrow+\infty}h(x)=\ell$, alors $\lim_{x\rightarrow+\infty}g(x)=\ell$.
\end{theoreme}

\begin{remarque}
On a, comme pour le théorème de comparaison précédent,  deux théorèmes analogues lorsque $x$ tend vers $-\infty$ et lorsque $x$ tend vers un réel $x_0$.\\
\end{remarque}

\begin{exemple}
Déterminons la limite en $-\infty$ de $f(x)=\dfrac{x\cos x}{x^2+1}$.\\
 La limite de $\cos x$ en $-\infty$ est indéterminée. Donc celle de $f(x)$ aussi.\\
 Cependant pour tout $x$ réel strictement négatif, $-1\leqslant\cos x\leqslant 1$ donc $x\leqslant x\cos x\leqslant -x$.\\
 Et en divisant membre à membre par $x^2+1>0$ on a : $\dfrac{x}{x^2+1}\leqslant \dfrac{x\cos x}{x^2+1}\leqslant \dfrac{-x}{x^2+1}$.\\[2mm]
Pour $x\in\mathbb{R}^*$, $\dfrac{x}{x^2+1}\hspace{-1mm} =\hspace{-1mm} \dfrac{1}{x+\tfrac{1}{x}}$. Or, $\lim_{x\rightarrow-\infty}\left(x+\frac{1}{x}\right)\hspace{-1mm} =\hspace{-1mm} -\infty$ donc $\lim_{x\rightarrow-\infty}\dfrac{x}{x^2+1}\hspace{-1mm} =\hspace{-1mm} \lim_{x\rightarrow-\infty}\dfrac{-x}{x^2+1}=0$.\\[2mm]
Donc, d'après le théorème des gendarmes, $\lim_{x\rightarrow-\infty}\dfrac{x\cos x}{x^2+1}=0$.
\end{exemple}

\section{Continuité d'une fonction}


\begin{definition}
Soit $f$ une fonction définie sur un intervalle $I$ et un réel $x_{0}\in I$.
\begin{itemize}
\item $f$ est \textbf{continue en} $\bolsymbol{x_{0}}$ si  $\lim_{x\to x_0}f(x)=f(x_0)$.
\item $f$ est \textbf{continue sur} $\boldsymbol{I}$ si, pour tout $x\in I$, $f$ est continue en $x$.
\end{itemize}
\end{definition}

\begin{remarque}
On peut considérer qu'une fonction est continue sur un intervalle $I$ si sa courbe représentative sur $I$ peut être tracée entièrement sans lever le crayon.
\end{remarque}

\begin{propriete}
\begin{itemize}
\item Les fonctions usuelles (affines, carré, inverse, racine carrée, valeur absolue) sont continues sur tout intervalle inclus dans  leur ensemble de définition.
\item Toute fonction construite algébriquement (par somme, produit, inverse ou composée) à partir de fonctions usuelles est continue  sur tout intervalle de son ensemble de définition.
\item On convient qu'une flèche oblique dans un tableau de variation traduit la continuité
et la stricte monotonie de la fonction sur l’intervalle considéré.
\item Une fonction dérivable sur un intervalle est continue sur cet intervalle.
  \end{itemize}
\end{propriete}

\begin{remarque}
Attention, la réciproque de cette dernière propriété est fausse.\\
Par exemple, la fonction valeur absolue $x\mapsto |x|$ est continue en 0 mais non dérivable en 0.
\end{remarque}

\newcommand{\lcro}{\textcolor{B2}{\textbf{\textsf{]}}}}
\newcommand{\rcro}{\textcolor{B2}{\textbf{\textsf{[}}}}

\begin{exemple}
Par convention,  une fonction est continue là où elle est tracée. S'il n'y a pas continuité en $x_0$ :
\begin{itemize}
\item le symbole \textcolor{B2}{$\bullet$} indique le point de la courbe de coordonnées $(x_0~;~f(x_0))$ ;
\item le symbole  \textcolor{white}{$\bullet$}\hspace{-5.5pt}\textcolor{B2}{$\boldsymbol{\circ}$} indique un  point qui n'appartient pas à la courbe mais dont l'ordonnée est égale à la limite à gauche ou à droite en $x_0$.
\end{itemize}

%\begin{colenumerate}{2}
%\item Soit  la fonction partie entière $f:x\mapsto \lfloor x\rfloor$.
%\begin{center}
%\psset{xunit=0.8cm,yunit=0.8cm,algebraic=true}
%\begin{pspicture*}(-3.5,-2.2)(3.5,2.8)
%\psgrid[subgriddiv=1,linewidth=0.5pt,gridcolor=A3,gridlabels=0pt](-4,-3)(4,3)
%\psaxes[linewidth=0.8pt,Dx=1,Dy=1,ticksize=-2pt]{->}(0,0)(-3.5,-2.2)(3.5,2.8)
%\rput[0](2.5,2.3){\textcolor{B2}{$\mathcal{C}_{f}$}}
%\pscircle[fillstyle=solid,linewidth=0.8pt,linecolor=B2,fillcolor=B2](-2,-2){.07}\psline[linecolor=B2,linewidth=0.8pt](-2,-2)(-1,-2)
%\rput[0](-1,-2){$\textcolor{white}{\bullet}$} \rput[0](-1,-2){$\textcolor{B2}{\boldsymbol{\circ}}$}
%\pscircle[fillstyle=solid,linewidth=0.8pt,linecolor=B2,fillcolor=B2](-1,-1){.07}\psline[linecolor=B2,linewidth=0.8pt](-1,-1)(0,-1)
%\rput[0](0,-1){$\textcolor{white}{\bullet}$} \rput[0](0,-1){$\textcolor{B2}{\boldsymbol{\circ}}$}
%\pscircle[fillstyle=solid,linewidth=0.8pt,linecolor=B2,fillcolor=B2](0,0){.07}\psline[linecolor=B2,linewidth=0.8pt](0,0)(1,0)
%\rput[0](1,0){$\textcolor{white}{\bullet}$} \rput[0](1,0){$\textcolor{B2}{\boldsymbol{\circ}}$}
%\pscircle[fillstyle=solid,linewidth=0.8pt,linecolor=B2,fillcolor=B2](1,1){.07}\psline[linecolor=B2,linewidth=0.8pt](1,1)(2,1)
%\rput[0](2,1){$\textcolor{white}{\bullet}$} \rput[0](2,1){$\textcolor{B2}{\boldsymbol{\circ}}$}
%\pscircle[fillstyle=solid,linewidth=0.8pt,linecolor=B2,fillcolor=B2](2,2){.07}\psline[linecolor=B2,linewidth=0.8pt](2,2)(3,2)
%\rput[0](3,2){$\textcolor{white}{\bullet}$} \rput[0](3,2){$\textcolor{B2}{\boldsymbol{\circ}}$}
%\end{pspicture*}
%\end{center}
%\item 
Soit  la fonction $f$ représentée ci-dessous.
\begin{center}
\psset{xunit=0.8cm,yunit=0.8cm,algebraic=true}
\begin{pspicture*}(-3.5,-2.2)(3.5,2.8)
\psgrid[subgriddiv=1,linewidth=0.5pt,gridcolor=A3,gridlabels=0pt](-4,-3)(4,3)
\psaxes[linewidth=0.8pt,Dx=1,Dy=1,ticksize=-2pt]{->}(0,0)(-3.5,-2.2)(3.5,2.8)
\rput[0](2.5,2.3){\textcolor{B2}{$\mathcal{C}_{f}$}}
\pscircle[fillstyle=solid,linewidth=0.8pt,linecolor=B2,fillcolor=B2](-3,-2){.07}
\pscircle[fillstyle=solid,linewidth=0.8pt,linecolor=B2,fillcolor=B2](0,-1){.07}
\pscircle[fillstyle=solid,linewidth=0.8pt,linecolor=B2,fillcolor=B2](3,2){.07}
\psline[linecolor=B2,linewidth=0.8pt](-3,-2)(0,-1)
\psline[linecolor=B2,linewidth=0.8pt](0,1)(3,2)
\rput[0](0,1){$\textcolor{white}{\bullet}$} \rput[0](0,1){$\textcolor{B2}{\boldsymbol{\circ}}$}
\end{pspicture*}
\end{center}
%\end{colenumerate}

  %\begin{enumerate}
  %\item En tout point d'abscisse $a\in\mathbb{Z}$, $\mathcal{C}_f$ présente un saut : on a $f(a)=a$ mais $\lim_{\substack{x\to a\\ x<a}} f(x)=a-1$. Ainsi, $f$ n'est pas continue en $a$ mais $f$ est continue sur tout intervalle $[a~;~a+1[$.
  %\item
	$f$ est " affine par morceaux ". $\mathcal{C}_f$ a un " saut " en 0 donc $f$ n'est pas continue sur $[-3~;~3]$ mais elle est continue sur $[-3~;~0]$ et $]0~;~3]$. En effet, on a $f(0)=-1$ mais $\lim_{\substack{x\to 0\\ x>0}} f(x)=1$.
    %\end{enumerate}
\end{exemple}



\section{Théorème des valeurs intermédiaires}



\begin{theoreme}[Cas général]
Soit $f$ une fonction définie sur un intervalle $I$  contenant deux réels $a$ et $b$ tels que $a<b$.\\
Si $f$ est continue sur  $[a~;~b]$, alors pour tout réel $k$ compris entre $f(a)$ et $f(b)$, il existe au moins un réel $c$ appartenant à $[a~;~b]$ tel que $f(c)=k$.
\end{theoreme}

\begin{remarque} $f$ prend au moins une fois toute valeur intermédiaire entre $f(a)$ et $f(b)$.\\
Autrement dit, l'équation $f(x)=k$ a  au moins une solution dans $[a~;~b]$ et, sur
$[a~;~b]$, la courbe représentative de $f$ coupe la droite d'équation $y=k$ en un point au moins.
\end{remarque}


\begin{exemple}
Soit $f$ la fonction définie sur $[0~;~6]$ par  $f(x)=\dfrac{x^3}{4}-\dfrac{9}{4}x^2+6x-3$.\\
\begin{minipage}{0.35\linewidth}
\begin{center}
\psset{xunit=.6cm,yunit=0.4cm,algebraic=true}
\begin{pspicture*}(-1,-3.4)(6.5,6.4)
\psgrid[subgriddiv=1,linewidth=0.5pt,gridcolor=A3,subgridcolor=A3,gridlabels=0pt](0,-4)(7,7)
\psaxes[linewidth=0.8pt,Dx=1,Dy=1,ticksize=-2pt]{->}(0,0)(-.6,-3.4)(6.5,6.4)
\uput[r](4.7,4.5){\textcolor{B2}{$\mathcal{C}_{f}$}}
\psplot[linecolor=B2,plotpoints=5000,linewidth=0.8pt]{0}{6}{(x^3-9*x^2+24*x-12)/4}
\pscircle[fillstyle=solid,linewidth=0.8pt,linecolor=B2,fillcolor=B2](0,-3){.07}
\pscircle[fillstyle=solid,linewidth=0.8pt,linecolor=B2,fillcolor=B2](6,6){.07}
\end{pspicture*}
\end{center}
\end{minipage}
\begin{minipage}{0.62\linewidth}
\begin{minipage}{0.45\linewidth}
On dresse le  tableau de variation de $f$.\\
$f$ admet pour minimum $-3$ et pour  maximum 6.\\
$f$ est continue sur $[0~;~6]$.
\end{minipage}\hspace{.5cm}
\begin{minipage}{0.3\linewidth}
\begin{center}
\begin{tikzpicture}[general, scale=0.75]
  \tkzTabInit[lgt=0.8, espcl=1.2, color, colorC=H4, colorV=H4, colorL=H4, colorT=A4, lw=0.3mm]
  {$x$ / 1, $f$ / 2}
  { $0$, $2$, $4$, $6$}
  \tkzTabVar{-/$-3$,+/$2$,-/$1$,+/$6$}
 \end{tikzpicture}
\end{center}
\end{minipage}\\[2mm]
Donc, d'après le théorème des valeurs intermédiaires, $f$ prend toutes les valeurs de  $[-3~;~6]$. En particulier, \mbox{l'équation} $f(x)=0$ a au moins une solution dans $[0~;~6]$.
\end{minipage}
\end{exemple}

\begin{theoreme}[Cas d'une fonction strictement monotone]
Soit $f$ une fonction définie sur un intervalle $I$ contenant deux réels $a$ et $b$ tels que $a<b$.\\
Si $f$ est continue et \textbf{strictement monotone} sur  $[a~;~b]$, alors pour tout réel $k$ compris entre $f(a)$ et $f(b)$, il existe \textbf{un unique} réel $c$ appartenant à $[a~;~b]$ tel que $f(c)=k$.
\end{theoreme}

\begin{exemple}
Reprenons la fonction $f:x\mapsto\dfrac{x^3}{4}-\dfrac{9}{4}x^2+6x-3$.\\
\begin{minipage}{0.23\linewidth}
\begin{center}
\psset{xunit=.8cm,yunit=0.6cm,algebraic=true}
\begin{pspicture*}(-.75,-3.5)(2.5,2.5)
\psgrid[subgriddiv=1,linewidth=0.5pt,gridcolor=A3,subgridcolor=A3,gridlabels=0pt](0,-4)(3,3)
\psaxes[linewidth=0.8pt,Dx=1,Dy=1,ticksize=-2pt]{->}(0,0)(-.75,-3.5)(2.5,2.5)
 \uput[ul](1,1){\textcolor{B2}{$\mathcal{C}_{f}$}}
\psplot[linecolor=B2,plotpoints=5000,linewidth=0.8pt]{0}{2}{(x^3-9*x^2+24*x-12)/4}
\pscircle[fillstyle=solid,linewidth=0.8pt,linecolor=B2,fillcolor=B2](0,-3){.07}
\pscircle[fillstyle=solid,linewidth=0.8pt,linecolor=B2,fillcolor=B2](2,2){.07}
\pscircle[fillstyle=solid,linewidth=0.8pt,linecolor=H2!60!black,fillcolor=Blanc](0.6647,0){.07}  \uput{9pt}[d](0.6647,0){\textcolor{H2!60!black}{$\alpha$}}
\end{pspicture*}
\end{center}
\end{minipage}
\begin{minipage}{0.27\linewidth}
\begin{center}
\begin{tikzpicture}[general, scale=0.8]
  \tkzTabInit[lgt=0.8, espcl=2, color, colorC=H4, colorV=H4, colorL=H4, colorT=A4, lw=0.3mm]
  {$x$ / 1, $f$ / 2}
  { $0$, $2$ }
  \tkzTabVar{-/$-3$,+/$2$}
  \tkzTabVal[draw]{1}{2}{0.5}{\textcolor{H2!60!black}{$\alpha$}}{$0$}
 \end{tikzpicture}
\end{center}
\end{minipage}
\begin{minipage}{0.5\linewidth}
Sur $[0~;~2]$, $f$ est continue, strictement croissante et admet pour minimum $-3$ et maximum $2$.\\
Donc, $f$ prend une fois, et une seule, toutes les valeurs intermédiaires entre $-3$ et 2.\\
En particulier, l'équation $f(x)=0$ a  une unique solution $\alpha$ entre 0 et 2.
\end{minipage}


\end{exemple}



\begin{remarque}
  Le théorème des valeurs intermédiaires s'applique aussi pour $f$
  continue sur un intervalle $I$ de type : $[a~;~b[$, $]a~;~b]$,
  $]a~;~b[$, $[a~;~+\infty[$, $]a~;~+\infty[$,
  $]-\infty~;~b]$ ou $]-\infty~;~b[$, $]-\infty\ ;\ +\infty[$.

  Si une borne $a$ ou $b$ de l'intervalle est ouverte, alors on
  remplace $f(a)$ ou $f(b)$ par la limite de $f$ en cette borne ; si
  une borne de l'intervalle est $\pm\infty$, alors on considère la
  limite de $f$ en $\pm\infty$.
\end{remarque}


\underline{Exercice : Exploiter le théorème des valeurs intermédiaires}\\


Le théorème des valeurs intermédiaires (T.V.I.) est utile pour prouver l'existence d'une \mbox{solution} d'une équation du type $f(x)=k$ et dénombrer ces solutions.Pour cela:
\begin{itemize}
\item On dresse le tableau de variation de la fonction $f$ ;
\item On applique le T.V.I. à chaque intervalle où la fonction est strictement monotone.
\end{itemize}

  

Dénombrer les solutions de l'équation (E) : $x^4+3x^3+x^2+1=0$.

  

\textbf{Correction}


$f:x\mapsto x^4+3x^3+x^2+1$ est une fonction polynôme de degré 4 dérivable sur $\mathbb{R}$.\\
$f'(x)=4x^3+9x^2+2x=x(4x^2+9x+2)=x(x+2)(4x+1)$ après factorisation du trinôme.\\
On établit alors le tableau de signes de $f'(x)$ et de variation de $f$ :
\begin{center}\newcommand{\valeur}{$\simeq1,02$}
 \begin{tikzpicture}[general, xscale=1.2, yscale=0.70, font=\fontsize{9}{10.5}\selectfont]
  \tkzTabInit[lgt=1, espcl=2.0, color, colorC=H4, colorV=H4, colorL=H4, colorT=A4, lw=0.3mm]
  {$x$ / 1, $f'(x)$ / 1, $f$ / 2}
  {$-\infty$, $-2$, $-\tfrac{1}{4}$, $0$, $+\infty$}
  \tkzTabLine{ ,-,z,+,z,-,z,+,}
  \tkzTabVar{+/$+\infty$,-/$-3$,+/\valeur,-/1,+/$+\infty$}
  \tkzTabVal[draw]{1}{2}{0.5}{\textcolor{H2!60!black}{$\alpha$}}{$0$}
 \tkzTabVal[draw]{2}{3}{0.5}{\textcolor{H2!60!black}{$\beta$}}{$0$}
 \end{tikzpicture}
 \end{center}

 \enlargethispage{12mm}

Sur $]-\infty~;~-2]$, $f$ est continue, strictement décroissante et : $\lim_{x\to-\infty}f(x)=+\infty$ ; $f(-2)=-3$.\\
Donc, d'après le T.V.I., l'équation (E) a une unique solution \textcolor{H2!60!black}{$\alpha$} inférieure à $-2$.\\
Sur $\left[-2~;~-\frac{1}{4}\right]$ , $f$ est continue, strictement croissante et : $f(-2)=-3$ ; $f\left(-\frac{1}{4}\right)\approx1,02>0$.\\
Donc, d'après le T.V.I., l'équation (E) a une unique solution \textcolor{H2!60!black}{$\beta$} comprise entre $-2$ et $-\frac{1}{4}$.\\
Sur $\left[-\frac{1}{4}~;~0\right]$ et $\left[0~;~+\infty\right[$, le minimum de $f$ est $1>0$ donc on n'y trouve pas de solution.\\
Conclusion : l'équation (E) admet exactement deux solutions dans $\mathbb{R}$.

\section{Fonction continue et suite convergente}

\begin{theoreme}
Soit une fonction $f$ définie et continue sur un intervalle $I$ et une suite $(u_n)$ telle que pour tout entier naturel $n$,on a : $u_n \in I$ et $u_{n+1}=f(u_n)$.\\
Si $(u_n)$ converge vers $l$ de $I$, alors $f(l)=l$.\\
Autrement dit : $l$ est solution de l'équation $f(x)=x$.
\end{theoreme}

\underline{Exemple} :\\
Soit la suite $(u_n)$ définie pour tout entier naturel $n$ par $u_0=0$ et $u_{n+1}=\sqrt{3u_n+4}$.\\

On peut démontrer par récurrence que $(u_n)$ est croissante et majorée par 4. (on l'admet sur cet exemple). Donc $(u_n)$ converge (théorème de convergence des suites monotones).

la fonction $f$ associée est f: x \mapsto \sqrt{3x+4}$.\\
 $f$ est continue sur $[-\frac{4}{3} ; +\infty[$ comme composée de fonctions continues sur cet intervalle.
($f=g\circ h$, avec $h : x \mapsto 3x+4$ et $g:x \mapsto \sqrt{x}$) 

Soit $l=\lim_{n \to +\infty} u_n$.
$l$ est solution de l'équation : $l=f(l) \iff l=\sqrt{3l+4}$.

Déterminons $l$.\\
$l$ existe ssi $3l+4 \geq 0$ ssi $l \geq -\frac{4}{3}$.

$l=\sqrt{3l+4} \Rightarrow l^2=3l+4 \iff l^2-3l-4=0$.\\
$\Delta=25$ . $l_1=-1$ et $l_2=4$.

Or, $u_0=0$ et $(u_n)$ est croissante donc la seule possibilité est $l_2=4$.

la limite de la suite $(u_n)$ est 4. 

\end{document}